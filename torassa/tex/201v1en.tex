
\documentclass[10pt,fleqn]{article}
%\documentclass[a4paper,10pt]{article}�
%\documentclass[letterpaper,10pt]{article}

%\usepackage[dvips]{geometry}
%\geometry{papersize={162.0mm,234.0mm}}
%\geometry{totalwidth=141.0mm,totalheight=198.0mm}
\usepackage[papersize={162.0mm,234.0mm},totalwidth=141.0mm,totalheight=198.0mm]{geometry}

\usepackage[english]{babel}
\usepackage[latin1]{inputenc}
%\usepackage[utf8]{inputenc}
%\usepackage[T1]{fontenc}
%\usepackage{ae,aecompl}
%\usepackage{mathptmx}
%\usepackage{pslatex}
%\usepackage{lmodern}
\usepackage{amsfonts}
\usepackage{amsmath}
\usepackage{amssymb}

%\frenchspacing

\usepackage{hyperref}
\hypersetup{colorlinks=true,linkcolor=black,urlcolor=blue,bookmarksopen=true}
\hypersetup{bookmarksnumbered=true,pdfstartview=FitH,pdfpagemode=UseOutlines}
%\hypersetup{bookmarksnumbered=true,pdfstartview=FitH,pdfpagemode=UseNone}
\hypersetup{pdftitle={Special Relativity: Types of Forces}}
\hypersetup{pdfauthor={A. Torassa}}

\setlength{\arraycolsep}{1.74pt}

\newcommand{\vet}{\smallskip}
\newcommand{\vek}{\bigskip \medskip}

%\ccl Attribution 4.0 International License
%\cct Attribution 4.0 International License
%\newcommand{\ccl}{${\large{\textcircled{\raisebox{+0.8pt}{{\scriptsize{{\textsf{cc}}}}}}}}$\;}
%\newcommand{\cct}{${\normalsize{\textcircled{\raisebox{+1.00pt}{{\tiny{{\texttt{cc}}}}}}}}$\;}

\newcommand{\med}{\raise.5ex\hbox{$\scriptstyle 1$}\kern-.15em/\kern-.09em\lower.25ex\hbox{$\scriptstyle 2$}}
\newcommand{\met}{\raise.5ex\hbox{$\scriptstyle 1$}\kern-.15em/\kern-.09em\lower.25ex\hbox{$\scriptstyle 2$}}

\begin{document}

\enlargethispage{+0.00em}

\addcontentsline{toc}{section}{Special Relativity: Types of Forces}

\begin{center}

{\LARGE Special Relativity: Types of Forces}

\vek

{\large A. Torassa}

\vek

\small

Creative Commons Attribution 4.0 License

\vet

\href{https://orcid.org/0000-0002-1389-247X}{\scriptsize{ORCID}} \S \ (2024) Buenos Aires
%\href{https://orcid.org/0000-0002-1389-247X}{\normalsize{\textsc{orcid}}} \S \ (2024) Buenos Aires

\vet

Argentina

\bigskip \bigskip

\parbox{102.90mm}{In special relativity, this paper presents four net forces, which can be applied in any massive or non-massive particle, and where the relationship between net force and special acceleration is as in Newton's second law \hbox {( that is,} the special acceleration of any massive or non-massive particle is always in the direction of the net force acting on the particle )}

\end{center}

\normalsize

\vspace{-1.20em}

\par \bigskip {\centering\subsection*{Introduction}}\addcontentsline{toc}{subsection}{Introduction}

\bigskip \smallskip

\noindent In special relativity, this paper is obtained starting from the essential definitions of intrinsic mass ( or invariant mass ) and relativistic factor ( or frequency factor ) for massive particles and non-massive particles.
\par \vspace{+0.60em}
\noindent The intrinsic mass $( \, m \, )$ and the relativistic factor $( \, f \, )$ of a massive particle, \hbox {are given by}:
\par \vspace{-0.60em}
\begin{eqnarray}
m ~\doteq~ m_o
\end{eqnarray}
\vspace{-0.90em}
\begin{eqnarray}
f ~\doteq~ \Big ( 1 - \dfrac{\mathbf{v} \cdot \mathbf{v}}{c^2} \hspace{+0.15em} \Big )^{\hspace{-0.24em}-\hspace{+0.03em}1/2}
\end{eqnarray}
\par \vspace{+0.60em}
\noindent where $( \, m_o \, )$ is the rest mass of the massive particle, $( \, \mathbf{v} \, )$ is the velocity of the massive particle, and $( \, c \, )$ is the speed of light in vacuum.
\par \vspace{+0.60em}
\noindent The intrinsic mass $( \, m \, )$ and the relativistic factor $( \, f \, )$ of a non-massive particle, \hbox {are given by}:
\par \vspace{-0.60em}
\begin{eqnarray}
m ~\doteq~ \dfrac{h \, \kappa}{c^2}
\end{eqnarray}
\vspace{-0.60em}
\begin{eqnarray}
f ~\doteq~ \dfrac{\nu}{\kappa}
\end{eqnarray}
\par \vspace{+0.60em}
\noindent where $( \hspace{+0.33em} h \hspace{+0.33em} )$ is the Planck constant, \hspace{+0.06em}$( \hspace{+0.30em} \nu \hspace{+0.30em} )$ is the frequency of the \hbox {non-massive} particle, $( \, \kappa \, )$ is a positive universal constant with dimension of frequency\hspace{-0.03em}, and $( \, c \, )$ is the speed of light in vacuum.
\par \vspace{+0.60em}
\noindent According to this paper, a massive particle $( \, m_o \ne 0 \, )$ is a particle with non-zero rest mass \hbox {( or a particle} whose speed $v$ in vacuum is less than $c$ ) and a non-massive particle $( \, m_o = 0 \, )$ is a particle with zero rest mass ( or a particle whose speed $v$ in vacuum is $c$ )
\par \vspace{+0.60em}
\noindent Note : The rest mass $( \, m_o \, )$ and the intrinsic mass $( \, m \, )$ are in general not additive, and the relativistic mass $( \, {\mathrm{m}} \, )$ of a particle ( massive or non-massive ) is given by : \hbox {\hspace{-0.09em}( ${\mathrm{m}} ~\doteq~ m \, f$ )}

\newpage

\par \bigskip {\centering\subsection*{The Einsteinian Kinematics}}\addcontentsline{toc}{subsection}{� 1 : The Einsteinian Kinematics}

\bigskip \smallskip

\noindent The special position $( \, \bar{\mathbf{r}} \, )$ the special velocity $( \, \bar{\mathbf{v}} \, )$ and the special acceleration $( \, \bar{\mathbf{a}} \, )$ of a particle \hbox {( massive or non-massive )} are given by:
\par \vspace{-0.30em}
\begin{eqnarray}
\bar{\mathbf{r}} \hspace{+0.12em}~\doteq~ \int f \, \mathbf{v} \; d\hspace{+0.012em}t
\end{eqnarray}
\vspace{-0.45em}
\begin{eqnarray}
\bar{\mathbf{v}} \hspace{-0.03em}~\doteq~ \dfrac{d\hspace{+0.036em}\bar{\mathbf{r}}}{d\hspace{+0.012em}t} ~=~ f \, \mathbf{v}
\end{eqnarray}
\vspace{-0.30em}
\begin{eqnarray}
\bar{\mathbf{a}} \hspace{+0.03em}~\doteq~ \dfrac{d\hspace{+0.021em}\bar{\mathbf{v}}}{d\hspace{+0.012em}t} ~=~ f \, \dfrac{d\hspace{+0.021em}\mathbf{v}}{d\hspace{+0.012em}t} + \dfrac{d\hspace{-0.12em}f}{d\hspace{+0.012em}t} \, \mathbf{v}
\end{eqnarray}
\par \vspace{+1.20em}
\noindent where $( \, f \, )$ is the relativistic factor of the particle, $( \, \mathbf{v} \, )$ is the velocity of the particle, and $( \, t \, )$ is the (\hspace{+0.15em}coordinate\hspace{+0.15em}) time.

\vspace{+0.60em}

\par \bigskip {\centering\subsection*{The Einsteinian Dynamics}}\addcontentsline{toc}{subsection}{� 1 : The Einsteinian Dynamics}

\bigskip \smallskip

\noindent If we consider a particle ( massive or non-massive ) with intrinsic mass $( \, m \, )$ then the \hbox {linear} momentum $( \, \mathbf{P} \, )$ of the particle, the angular momentum $( \, \mathbf{L} \, )$ of the particle, the net \hbox {Einsteinian} force $( \, \mathbf{F}\mathrm{\scriptscriptstyle{E}} \, )$ acting on the particle, the work $( \, \mathrm{W} \, )$ done by the net Einsteinian force acting on the particle, and the kinetic energy $( \, \mathrm{K} \, )$ of the particle, are given by:
\par \vspace{-0.30em}
\begin{eqnarray}
\mathbf{P} \hspace{-0.06em}~\doteq~ m \, \bar{\mathbf{v}} ~=~ m \, f \, \mathbf{v}
\end{eqnarray}
\vspace{-0.30em}
\begin{eqnarray}
\mathbf{L} \hspace{+0.03em}~\doteq~ \mathbf{r} \times \mathbf{P} ~=~ m \; \mathbf{r} \times \bar{\mathbf{v}} ~=~ m \, f \: \mathbf{r} \times \mathbf{v}
\end{eqnarray}
\vspace{-0.30em}
\hypertarget{efe}{}
\begin{eqnarray}
\mathbf{F}\mathrm{\scriptscriptstyle{E}} ~=~ \dfrac{d\hspace{+0.045em}\mathbf{P}}{d\hspace{+0.012em}t} ~=~ m \, \bar{\mathbf{a}} ~=~ m \, \bigg [ \, f \, \dfrac{d\hspace{+0.021em}\mathbf{v}}{d\hspace{+0.012em}t} + \dfrac{d\hspace{-0.12em}f}{d\hspace{+0.012em}t} \, \mathbf{v} \, \bigg ]
\end{eqnarray}
\vspace{-0.15em}
\begin{eqnarray}
\mathrm{W} \hspace{-0.312em}~\doteq~ \hspace{-0.36em} \int_{\scriptscriptstyle 1}^{\hspace{+0.09em}{\scriptscriptstyle 2}} \mathbf{F}\mathrm{\scriptscriptstyle{E}} \cdot d\hspace{+0.036em}\mathbf{r} ~=~ \hspace{-0.36em} \int_{\scriptscriptstyle 1}^{\hspace{+0.09em}{\scriptscriptstyle 2}} \dfrac{d\hspace{+0.045em}\mathbf{P}}{d\hspace{+0.012em}t} \cdot d\hspace{+0.036em}\mathbf{r} ~=~ \Delta \, \mathrm{K}
\end{eqnarray}
\vspace{-0.30em}
\begin{eqnarray}
\mathrm{K} \hspace{-0.06em}~\doteq~ m \, f \, c^2
\end{eqnarray}
\par \vspace{+0.90em}
\noindent where $( \: f, \: \mathbf{r}, \: \mathbf{v}, \: \bar{\mathbf{v}}, \: \bar{\mathbf{a}} \: )$ are the relativistic factor, the position, the velocity, the special velocity and the special acceleration of the particle, $( \, t \, )$ is the (\hspace{+0.15em}coordinate\hspace{+0.15em}) time, and $( \, c \, )$ is the speed of light in vacuum. The kinetic energy $( \, \mathrm{K}_o \, )$ of a massive particle at rest is $( \, m_o \, c^2 \, )$ since in this dynamics the relativistic energy $( \, \mathrm{E} \,\doteq\, m_o \, c^2 \, (\,f - 1\,) + m_o \, c^2 \, )$ and the kinetic energy $( \, \mathrm{K} \,\doteq\, m \, f \, c^2 \, )$ are the same $( \, \mathrm{E} \,=\, \mathrm{K} \, )$ \hyperlink{ref}{[\,1\,]}
\par \vspace{+0.60em}
\noindent {\small Note :} {\footnotesize $\mathrm{E}^2 - \mathbf{P}^2 c^2 = m^2 \, f^2 \, c^4 \, (\hspace{+0.12em} 1 - \mathbf{v}^2/c^2 \hspace{+0.15em})$} {\small [ in massive particle :} {\footnotesize $f^2 \, (\hspace{+0.12em} 1 - \mathbf{v}^2/c^2 \hspace{+0.15em}) = 1$ \hspace{+0.27em}$\rightarrow$\hspace{+0.27em} \hbox {{$\mathrm{E}^2 - \mathbf{P}^2 c^2 = {m_o}^2 c^4$}} \,and\, $m \ne 0$} {\small ]} \,{\footnotesize\&}\, {\small [ in non-massive particle :} {\footnotesize $\mathbf{v}^2 = c^2$ \hspace{+0.27em}$\rightarrow$\hspace{+0.27em} $(\hspace{+0.12em} 1 - \mathbf{v}^2/c^2 \hspace{+0.15em}) = 0$ \hspace{+0.27em}$\rightarrow$\hspace{+0.27em} \hbox {{$\mathrm{E}^2 - \mathbf{P}^2 c^2 = 0$ \:and\: $m \ne 0$} {\small ]}}} {\footnotesize In special relativity there are 3 types of masses: rest mass ($m_o$) intrinsic mass ($m$) and relativistic mass (${\mathrm{m}}$)}

\newpage

\par \bigskip {\centering\subsection*{The Newtonian Kinematics}}\addcontentsline{toc}{subsection}{� 2 : The Newtonian Kinematics}

\bigskip \smallskip

\noindent The special position $( \, \bar{\mathbf{r}} \, )$ the special velocity $( \, \bar{\mathbf{v}} \, )$ and the special acceleration $( \, \bar{\mathbf{a}} \, )$ of a particle \hbox {( massive or non-massive )} are given by:
\par \vspace{-0.30em}
\begin{eqnarray}
\bar{\mathbf{r}} \hspace{+0.12em}~\doteq~ \mathbf{r}
\end{eqnarray}
\vspace{-0.45em}
\begin{eqnarray}
\bar{\mathbf{v}} \hspace{-0.03em}~\doteq~ \dfrac{d\hspace{+0.036em}\bar{\mathbf{r}}}{d\hspace{+0.012em}t} ~=~ \mathbf{v}
\end{eqnarray}
\vspace{-0.30em}
\begin{eqnarray}
\bar{\mathbf{a}} \hspace{+0.03em}~\doteq~ \dfrac{d\hspace{+0.021em}\bar{\mathbf{v}}}{d\hspace{+0.012em}t} ~=~ \mathbf{a}
\end{eqnarray}
\par \vspace{+1.20em}
\noindent where $( \, \mathbf{r} \, )$ is the position of the particle, $( \, \mathbf{v} \, )$ is the velocity of the particle, $( \, \mathbf{a} \, )$ is the acceleration of the particle, and $( \, t \, )$ is the (\hspace{+0.15em}coordinate\hspace{+0.15em}) time.

\vspace{+0.60em}

\par \bigskip {\centering\subsection*{The Newtonian Dynamics}}\addcontentsline{toc}{subsection}{� 2 : The Newtonian Dynamics}

\bigskip \smallskip

\noindent If we consider a particle ( massive or non-massive ) with intrinsic mass $( \, m \, )$ then the \hbox {linear} momentum $( \, \mathbf{P} \, )$ of the particle, the angular momentum $( \, \mathbf{L} \, )$ of the particle, the net \hbox {Newtonian} force $( \, \mathbf{F}\mathrm{\scriptscriptstyle{N}} \, )$ acting on the particle, the work $( \, \mathrm{W} \, )$ done by the net Newtonian force acting on the particle, and the kinetic energy $( \, \mathrm{K} \, )$ of the particle, are given by:
\par \vspace{-0.30em}
\begin{eqnarray}
\mathbf{P} \hspace{-0.06em}~\doteq~ m \, \bar{\mathbf{v}} ~=~ m \, \mathbf{v}
\end{eqnarray}
\vspace{-0.30em}
\begin{eqnarray}
\mathbf{L} \hspace{+0.03em}~\doteq~ \mathbf{r} \times \mathbf{P} ~=~ m \; \mathbf{r} \times \bar{\mathbf{v}} ~=~ m \; \mathbf{r} \times \mathbf{v}
\end{eqnarray}
\vspace{-0.30em}
\hypertarget{efn}{}
\begin{eqnarray}
\mathbf{F}\mathrm{\scriptscriptstyle{N}} ~=~ \dfrac{d\hspace{+0.045em}\mathbf{P}}{d\hspace{+0.012em}t} ~=~ m \, \bar{\mathbf{a}} ~=~ m \, \mathbf{a}
\end{eqnarray}
\vspace{-0.15em}
\begin{eqnarray}
\mathrm{W} \hspace{-0.312em}~\doteq~ \hspace{-0.36em} \int_{\scriptscriptstyle 1}^{\hspace{+0.09em}{\scriptscriptstyle 2}} \mathbf{F}\mathrm{\scriptscriptstyle{N}} \cdot d\hspace{+0.036em}\mathbf{r} ~=~ \hspace{-0.36em} \int_{\scriptscriptstyle 1}^{\hspace{+0.09em}{\scriptscriptstyle 2}} \dfrac{d\hspace{+0.045em}\mathbf{P}}{d\hspace{+0.012em}t} \cdot d\hspace{+0.036em}\mathbf{r} ~=~ \Delta \, \mathrm{K}
\end{eqnarray}
\vspace{-0.30em}
\begin{eqnarray}
\mathrm{K} \hspace{-0.06em}~\doteq~ \med \hspace{+0.36em} m \: ( \mathbf{v} \cdot \mathbf{v} )
\end{eqnarray}
\par \vspace{+0.90em}
\noindent where $( \: \mathbf{r}, \: \mathbf{v}, \: \mathbf{a}, \: \bar{\mathbf{v}}, \: \bar{\mathbf{a}} \: )$ are the position, the velocity, the acceleration, the special velocity and the special acceleration of the particle, $( \, t \, )$ is the (\hspace{+0.15em}coordinate\hspace{+0.15em}) time, and $( \, c \, )$ is the speed of light in vacuum. The kinetic energy $( \, \mathrm{K}_o \, )$ of a massive particle at rest is zero, and the ordinary acceleration $( \, \mathbf{a} \, )$ of a massive or non-massive particle is also always in the direction of the net Newtonian force $( \, \mathbf{F}\mathrm{\scriptscriptstyle{N}} \, )$ acting on the particle.
\par \vspace{+0.60em}
\noindent In special relativity, the net Newtonian force $( \, \mathbf{F}\mathrm{\scriptscriptstyle{N}} \, )$ acting on a massive or non-massive particle, is given by : $\mathbf{F}\mathrm{\scriptscriptstyle{N}} ~\doteq~ \mathbf{N}^{\scriptscriptstyle -1} \cdot \mathbf{F}\mathrm{\scriptscriptstyle{E}}$ , where $( \, \mathbf{N} \, )$ is the Newton tensor, and $( \, \mathbf{F}\mathrm{\scriptscriptstyle{E}} \, )$ is the net Einsteinian force acting on the massive or non-massive particle \hyperlink{ref}{[\,2\,]}

\newpage

\par \bigskip {\centering\subsection*{The Poincarian Kinematics}}\addcontentsline{toc}{subsection}{� 3 : The Poincarian Kinematics}

\bigskip \smallskip

\noindent The special position $( \, \bar{\mathbf{r}} \, )$ the special velocity $( \, \bar{\mathbf{v}} \, )$ and the special acceleration $( \, \bar{\mathbf{a}} \, )$ of a particle \hbox {( massive or non-massive )} are given by:
\par \vspace{-0.30em}
\begin{eqnarray}
\bar{\mathbf{r}} \hspace{+0.12em}~\doteq~ \mathbf{r}
\end{eqnarray}
\vspace{-0.45em}
\begin{eqnarray}
\bar{\mathbf{v}} \hspace{-0.03em}~\doteq~ \dfrac{d\hspace{+0.036em}\bar{\mathbf{r}}}{d\hspace{+0.012em}\tau} ~=~ f \, \mathbf{v}
\end{eqnarray}
\vspace{-0.30em}
\begin{eqnarray}
\bar{\mathbf{a}} \hspace{+0.03em}~\doteq~ \dfrac{d\hspace{+0.021em}\bar{\mathbf{v}}}{d\hspace{+0.012em}\tau} ~=~ f \, \bigg [ \, f \, \dfrac{d\hspace{+0.021em}\mathbf{v}}{d\hspace{+0.012em}t} + \dfrac{d\hspace{-0.12em}f}{d\hspace{+0.012em}t} \, \mathbf{v} \, \bigg ]
\end{eqnarray}
\par \vspace{+1.20em}
\noindent where $( \, f \, )$ is the relativistic factor of the particle, $( \, \mathbf{r} \, )$ is the position of the particle, $( \, \mathbf{v} \, )$ is the velocity of the particle, and $( \, \tau \, )$ is the proper time of the particle ( Note : $d\tau \,=\, f^{\scriptscriptstyle -1} \: dt$ )

\vspace{+0.60em}

\par \bigskip {\centering\subsection*{The Poincarian Dynamics}}\addcontentsline{toc}{subsection}{� 3 : The Poincarian Dynamics}

\bigskip \smallskip

\noindent If we consider a particle ( massive or non-massive ) with intrinsic mass $( \, m \, )$ then the \hbox {linear} momentum $( \, \mathbf{P} \, )$ of the particle, the angular momentum $( \, \mathbf{L} \, )$ of the particle, the net \hbox {Poincarian} force $( \, \mathbf{F}\mathrm{\scriptscriptstyle{P}} \, )$ acting on the particle, the work $( \, \mathrm{W} \, )$ done by the net Poincarian force acting on the particle, and the kinetic energy $( \, \mathrm{K} \, )$ of the particle, are given by:
\par \vspace{-0.30em}
\begin{eqnarray}
\mathbf{P} \hspace{-0.06em}~\doteq~ m \, \bar{\mathbf{v}} ~=~ m \, f \, \mathbf{v}
\end{eqnarray}
\vspace{-0.30em}
\begin{eqnarray}
\mathbf{L} \hspace{+0.03em}~\doteq~ \mathbf{r} \times \mathbf{P} ~=~ m \; \mathbf{r} \times \bar{\mathbf{v}} ~=~ m \, f \: \mathbf{r} \times \mathbf{v}
\end{eqnarray}
\vspace{-0.30em}
\hypertarget{efp}{}
\begin{eqnarray}
\mathbf{F}\mathrm{\scriptscriptstyle{P}} ~=~ \dfrac{d\hspace{+0.045em}\mathbf{P}}{d\hspace{+0.012em}\tau} ~=~ m \, \bar{\mathbf{a}} ~=~ m \, f \, \bigg [ \, f \, \dfrac{d\hspace{+0.021em}\mathbf{v}}{d\hspace{+0.012em}t} + \dfrac{d\hspace{-0.12em}f}{d\hspace{+0.012em}t} \, \mathbf{v} \, \bigg ]
\end{eqnarray}
\vspace{-0.15em}
\begin{eqnarray}
\mathrm{W} \hspace{-0.312em}~\doteq~ \hspace{-0.36em} \int_{\scriptscriptstyle 1}^{\hspace{+0.09em}{\scriptscriptstyle 2}} f^{\scriptscriptstyle -1} \: \mathbf{F}\mathrm{\scriptscriptstyle{P}} \cdot d\hspace{+0.036em}\mathbf{r} ~=~ \hspace{-0.36em} \int_{\scriptscriptstyle 1}^{\hspace{+0.09em}{\scriptscriptstyle 2}} f^{\scriptscriptstyle -1} \: \dfrac{d\hspace{+0.045em}\mathbf{P}}{d\hspace{+0.012em}\tau} \cdot d\hspace{+0.036em}\mathbf{r} ~=~ \Delta \, \mathrm{K}
\end{eqnarray}
\vspace{-0.30em}
\begin{eqnarray}
\mathrm{K} \hspace{-0.06em}~\doteq~ m \, f \, c^2
\end{eqnarray}
\par \vspace{+0.90em}
\noindent where $( \: f, \: \mathbf{r}, \: \mathbf{v}, \: \tau, \: \bar{\mathbf{v}}, \: \bar{\mathbf{a}} \: )$ are the relativistic factor, the position, the velocity, the proper time, the special velocity and the special acceleration of the particle, and $( \, c \, )$ is the speed of light in vacuum. The kinetic energy $( \, \mathrm{K}_o \, )$ of a massive particle at rest is $( \, m_o \, c^2 \, )$ since also in this dynamics the relativistic energy $( \, \mathrm{E} \,\doteq\, m_o \, c^2 \, (\,f - 1\,) + m_o \, c^2 \, )$ and the kinetic energy $( \, \mathrm{K} \,\doteq\, m \, f \, c^2 \, )$ are the same $( \, \mathrm{E} \,=\, \mathrm{K} \, )$
\par \vspace{+0.60em}
\noindent In special relativity, the net Poincarian force $( \, \mathbf{F}\mathrm{\scriptscriptstyle{P}} \, )$ acting on a massive or non-massive particle is given by : $\mathbf{F}\mathrm{\scriptscriptstyle{P}} ~\doteq~ f \hspace{+0.36em} \mathbf{F}\mathrm{\scriptscriptstyle{E}}$ , where $( \, f \, )$ is the relativistic factor of the particle, and $( \, \mathbf{F}\mathrm{\scriptscriptstyle{E}} \, )$ is the net Einsteinian force acting on the massive or non-massive particle \hyperlink{ref}{[\,3\,]}

\newpage

\par \bigskip {\centering\subsection*{The M{\o}llerian Kinematics}}\addcontentsline{toc}{subsection}{� 4 : The M{\o}llerian Kinematics}

\bigskip \smallskip

\noindent The special position $( \, \bar{\mathbf{r}} \, )$ the special velocity $( \, \bar{\mathbf{v}} \, )$ and the special acceleration $( \, \bar{\mathbf{a}} \, )$ of a particle \hbox {( massive or non-massive )} are given by:
\par \vspace{-0.30em}
\begin{eqnarray}
\bar{\mathbf{r}} \hspace{+0.12em}~\doteq~ \int \mathbf{v} \; d\hspace{+0.012em}\tau
\end{eqnarray}
\vspace{-0.45em}
\begin{eqnarray}
\bar{\mathbf{v}} \hspace{-0.03em}~\doteq~ \dfrac{d\hspace{+0.036em}\bar{\mathbf{r}}}{d\hspace{+0.012em}\tau} ~=~ \mathbf{v}
\end{eqnarray}
\vspace{-0.30em}
\begin{eqnarray}
\bar{\mathbf{a}} \hspace{+0.03em}~\doteq~ \dfrac{d\hspace{+0.021em}\bar{\mathbf{v}}}{d\hspace{+0.012em}\tau} ~=~ f \, \mathbf{a}
\end{eqnarray}
\par \vspace{+1.20em}
\noindent where $( \, f \, )$ is the relativistic factor of the particle, $( \, \mathbf{r}, \: \mathbf{v}, \: \mathbf{a} \, )$ are the position, the velocity and the acceleration of the particle, \hspace{-0.03em}and \hspace{-0.03em}$( \, \tau \, )$ is the proper time of the particle \hbox {{\small ( Note : $d\tau \,=\, f^{\scriptscriptstyle -1} \: dt$ )}}

\vspace{+0.60em}

\par \bigskip {\centering\subsection*{The M{\o}llerian Dynamics}}\addcontentsline{toc}{subsection}{� 4 : The M{\o}llerian Dynamics}

\bigskip \smallskip

\noindent If we consider a particle ( massive or non-massive ) with intrinsic mass $( \, m \, )$ then the \hbox {linear} momentum $( \, \mathbf{P} \, )$ of the particle, the angular momentum $( \, \mathbf{L} \, )$ of the particle, the net \hbox {M{\o}llerian} force $( \, \mathbf{F}\mathrm{\scriptscriptstyle{M}} \, )$ acting on the particle, the work $( \, \mathrm{W} \, )$ done by the net M{\o}llerian force acting on the particle, and the kinetic energy $( \, \mathrm{K} \, )$ of the particle, are given by:
\par \vspace{-0.30em}
\begin{eqnarray}
\mathbf{P} \hspace{-0.06em}~\doteq~ m \, \bar{\mathbf{v}} ~=~ m \, \mathbf{v}
\end{eqnarray}
\vspace{-0.30em}
\begin{eqnarray}
\mathbf{L} \hspace{+0.03em}~\doteq~ \mathbf{r} \times \mathbf{P} ~=~ m \; \mathbf{r} \times \bar{\mathbf{v}} ~=~ m \; \mathbf{r} \times \mathbf{v}
\end{eqnarray}
\vspace{-0.30em}
\hypertarget{efm}{}
\begin{eqnarray}
\mathbf{F}\mathrm{\scriptscriptstyle{M}} ~=~ \dfrac{d\hspace{+0.045em}\mathbf{P}}{d\hspace{+0.012em}\tau} ~=~ m \, \bar{\mathbf{a}} ~=~ m \, f \, \mathbf{a}
\end{eqnarray}
\vspace{-0.15em}
\begin{eqnarray}
\mathrm{W} \hspace{-0.312em}~\doteq~ \hspace{-0.36em} \int_{\scriptscriptstyle 1}^{\hspace{+0.09em}{\scriptscriptstyle 2}} f^{\scriptscriptstyle -1} \: \mathbf{F}\mathrm{\scriptscriptstyle{M}} \cdot d\hspace{+0.036em}\mathbf{r} ~=~ \hspace{-0.36em} \int_{\scriptscriptstyle 1}^{\hspace{+0.09em}{\scriptscriptstyle 2}} f^{\scriptscriptstyle -1} \: \dfrac{d\hspace{+0.045em}\mathbf{P}}{d\hspace{+0.012em}\tau} \cdot d\hspace{+0.036em}\mathbf{r} ~=~ \Delta \, \mathrm{K}
\end{eqnarray}
\vspace{-0.30em}
\begin{eqnarray}
\mathrm{K} \hspace{-0.06em}~\doteq~ \med \hspace{+0.36em} m \: ( \mathbf{v} \cdot \mathbf{v} )
\end{eqnarray}
\par \vspace{+0.90em}
\noindent where $( \: f, \: \mathbf{r}, \: \mathbf{v}, \: \mathbf{a}, \: \tau, \: \bar{\mathbf{v}}, \: \bar{\mathbf{a}} \: )$ are the relativistic factor, the position, the velocity, the acceleration, the proper time, the special velocity and the special acceleration of the particle, and $( \, c \, )$ is the speed of light in vacuum. The kinetic energy $( \, \mathrm{K}_o \, )$ of a massive particle at rest is zero, and the ordinary acceleration $( \, \mathbf{a} \, )$ of a massive or non-massive particle is also always in the direction of the net M{\o}llerian force $( \, \mathbf{F}\mathrm{\scriptscriptstyle{M}} \, )$ acting on the particle.
\par \vspace{+0.60em}
\noindent In special relativity, the net M{\o}llerian force $( \, \mathbf{F}\mathrm{\scriptscriptstyle{M}} \, )$ acting on a massive or non-massive particle is given by : $\mathbf{F}\mathrm{\scriptscriptstyle{M}} ~\doteq~ \mathbf{M} \,\cdot\, \mathbf{F}\mathrm{\scriptscriptstyle{E}}$ , where $( \, \mathbf{M} \, )$ is the M{\o}ller tensor, and $( \, \mathbf{F}\mathrm{\scriptscriptstyle{E}} \, )$ is the net Einsteinian force acting on the massive or non-massive particle \hyperlink{an1}{\hbox {(\hspace{+0.120em}see {\small A}nnex {\small I}\hspace{+0.120em})}} \hyperlink{ref}{[\,4\,]}

\newpage

\par \bigskip {\centering\subsection*{General Observations}}\addcontentsline{toc}{subsection}{General Observations}

\bigskip \smallskip

\noindent In special relativity, the net forces $[ \: \mathbf{F}\mathrm{\scriptscriptstyle{N}}, \mathbf{F}\mathrm{\scriptscriptstyle{P}}, \mathbf{F}\mathrm{\scriptscriptstyle{M}} \: ]$ are valid since these net forces are obtained from the net Einsteinian force $[ \: \mathbf{F}\mathrm{\scriptscriptstyle{E}} \: ]$
\par \medskip \smallskip
\noindent Therefore, the net forces $[ \: \mathbf{F}\mathrm{\scriptscriptstyle{E}}, \mathbf{F}\mathrm{\scriptscriptstyle{N}}, \mathbf{F}\mathrm{\scriptscriptstyle{P}}, \mathbf{F}\mathrm{\scriptscriptstyle{M}} \: ]$ can be applied in any inertial reference frame.
\par \medskip \smallskip
\noindent The special acceleration $( \, \bar{\mathbf{a}} \, )$ of a particle (\hspace{+0.12em}massive or non-massive\hspace{+0.12em}) is always in the direction of the net forces $[ \: \mathbf{F}\mathrm{\scriptscriptstyle{E}}, \mathbf{F}\mathrm{\scriptscriptstyle{N}}, \mathbf{F}\mathrm{\scriptscriptstyle{P}}, \mathbf{F}\mathrm{\scriptscriptstyle{M}} \: ]$ acting on the particle (\hspace{+0.12em}as in Newton's second law\hspace{+0.12em})
\par \medskip \smallskip
\noindent Additionally, the ordinary acceleration $( \, \mathbf{a} \, )$ of a particle (\hspace{+0.12em}massive or non-massive\hspace{+0.12em}) is also always in the direction of the net forces $[ \: \mathbf{F}\mathrm{\scriptscriptstyle{N}} \, , \mathbf{F}\mathrm{\scriptscriptstyle{M}} \: ]$ acting on the particle (\hspace{+0.12em}exactly as in Newton's second law\hspace{+0.12em}) \hbox {( {\small Note} : $\mathbf{F}\mathrm{\scriptscriptstyle{N}} = f^{\scriptscriptstyle -1} \, \mathbf{F}\mathrm{\scriptscriptstyle{M}} = f^{\scriptscriptstyle -1} \, \mathbf{M} \cdot \mathbf{F}\mathrm{\scriptscriptstyle{E}}$ )}
\par \medskip \smallskip
\noindent The net forces $[ \: \mathbf{F}\mathrm{\scriptscriptstyle{E}}, \mathbf{F}\mathrm{\scriptscriptstyle{N}}, \mathbf{F}\mathrm{\scriptscriptstyle{P}}, \mathbf{F}\mathrm{\scriptscriptstyle{M}} \: ]$ are three-forces ( that is, they are three-dimensional \hbox {vectors )}
\par \medskip \smallskip
\noindent On the other hand, the net Minkowskian four-force $[ \: \mathbf{\overline{F}}\mathrm{\scriptscriptstyle{M}} \: ]$ is obtained from the four-momentum and the proper time of a massive particle. In addition, the net Einsteinian four-force $[ \: \mathbf{\overline{F}}\mathrm{\scriptscriptstyle{E}} \: ]$ can be obtained from the four-momentum and the (\hspace{+0.15em}coordinate\hspace{+0.15em}) time of a massive particle \hbox {( {\small Note} : $\mathbf{\overline{F}}\mathrm{\scriptscriptstyle{E}} = ( \; ( \, d\hspace{+0.012em}\mathrm{E}\hspace{+0.12em}/\hspace{+0.03em}dt \, ) \: c^{\scriptscriptstyle -1} \, , \, \mathbf{F}\mathrm{\scriptscriptstyle{E}} \; )$\hspace{+0.51em}and\hspace{+0.51em}$ \mathbf{\overline{F}}\mathrm{\scriptscriptstyle{M}} = f \hspace{+0.36em} \mathbf{\overline{F}}\mathrm{\scriptscriptstyle{E}}$\hspace{+0.36em})\hspace{+0.48em}[\hspace{+0.36em}see : \hyperlink{apa}{{\small A}ppendix {\small A}\hspace{+0.120em}} and \hyperlink{apb}{{\small A}ppendix {\small B}\hspace{+0.120em}}\hspace{+0.36em}]}
\par \medskip \smallskip
\noindent In special relativity, there are three types of masses that are compatible with each other : the rest mass $( \, m_o \, )$ the intrinsic mass $( \, m \, )$ and the relativistic mass $( \, {\mathrm{m}} \, )$ \hbox {( the intrinsic} mass $( \, m \, )$ is an invariant mass that can be applied to massive and non-massive particles )
\par \medskip \smallskip
\noindent In the Poincarian dynamics, the definition of work $( \, \mathrm{W} \, )$ is modified so that the magnitudes $( \, \mathbf{P}, \mathrm{K} \, )$ match the magnitudes $( \, \mathbf{P}, \mathrm{K} \, )$ of the Einsteinian dynamics. 
\par \medskip \smallskip
\noindent In the M{\o}llerian dynamics, the definition of work $( \, \mathrm{W} \, )$ is modified so that the magnitudes $( \, \mathbf{P}, \mathrm{K} \, )$ match the magnitudes $( \, \mathbf{P}, \mathrm{K} \, )$ of the Newtonian dynamics. 
\par \medskip \smallskip
\noindent Additionally, in relativistic elastic collisions ( or relativistic elastic shocks ) between massive and{\small /}or non-massive particles of an isolated system, the magnitudes $(\hspace{+0.12em}\mathbf{P} = \sum \, m_i \, f_i \, \mathbf{v}_i\hspace{+0.12em})$ and \hbox {$(\hspace{+0.12em}\mathrm{K} = \sum \, m_i \, f_i \, c^2\hspace{+0.12em})$} are conserved [\hspace{+0.12em}and the net Einsteinian force $(\hspace{+0.12em}\mathbf{F}\mathrm{\scriptscriptstyle{E}} = d\mathbf{P}/dt\hspace{+0.12em})$ is always \hbox {zero\hspace{+0.12em}]}

\vspace{+0.60em}

\par \bigskip {\centering\subsection*{References \& Bibliography}}\addcontentsline{toc}{subsection}{References \& Bibliography}\hypertarget{ref}{}

\bigskip \smallskip

\par \noindent [\,1\,] \textbf{A. Tobla}, A Reformulation of Special Relativity, (2024)\hspace{+0.09em}.\hspace{+0.09em}(\hspace{+0.09em}\href{https://doi.org/10.5281/zenodo.11466228}{\texttt{doi}}\hspace{+0.09em})
\bigskip \smallskip
\par \noindent [\,2\,] \textbf{A. Blato}, Special Relativity \& Newton's Second Law, (2016)\hspace{+0.09em}.\hspace{+0.09em}(\hspace{+0.09em}\href{http://dx.doi.org/10.13140/RG.2.1.4520.7441}{\texttt{doi}}\hspace{+0.09em})
\bigskip \smallskip
\par \noindent [\,3\,] \textbf{A. Blato}, A New Dynamics in Special Relativity, (2016)\hspace{+0.09em}.\hspace{+0.09em}(\hspace{+0.09em}\href{http://dx.doi.org/10.13140/RG.2.1.3341.0961}{\texttt{doi}}\hspace{+0.09em})
\bigskip \smallskip
\par \noindent [\,4\,] \textbf{C. M{\o}ller}, The Theory of Relativity, (1952)\hspace{+0.09em}.
\bigskip \smallskip
\par \noindent [\hspace{+0.063em}{\small A}\hspace{+0.063em}] \textbf{W\hspace{-0.18em}. Pauli}, Theory of Relativity, (1958)\hspace{+0.09em}.
\bigskip \smallskip
\par \noindent [\hspace{+0.093em}{\small B}\hspace{+0.093em}] \textbf{A. French}, Special Relativity, (1968)\hspace{+0.09em}.

\newpage

\par \bigskip {\centering\subsection*{Annex I}}\hypertarget{an1}{}

\par \bigskip {\centering\subsection*{The M{\o}ller Tensor}}\addcontentsline{toc}{subsection}{Annex I : The M{\o}ller Tensor}

\bigskip \smallskip

\noindent The M{\o}ller tensor ($\mathbf{M}$) and the net M{\o}llerian force ($\mathbf{F}\mathrm{\scriptscriptstyle{M}}$) can be obtained from the net Einsteinian force ($\mathbf{F}\mathrm{\scriptscriptstyle{E}}$) acting on a massive particle with rest mass ($m_o$)
\par \vspace{+0.30em}
\begin{eqnarray}
m_o \left [ \; \frac{\mathbf{a}}{\left(1 - \frac{v^2}{c^2}\right)^{\hspace{-0.12em}{1/2}}} + \frac{(\mathbf{a} \cdot \mathbf{v}) \: \mathbf{v}}{c^2\hspace{-0.15em}\left(1 - \frac{v^2}{c^2}\right)^{\hspace{-0.12em}{3/2}}} \; \right ] ~=~ \mathbf{F}\mathrm{\scriptscriptstyle{E}}
\end{eqnarray}
\vspace{+0.60em}
\begin{eqnarray}
m_o \left [ \; \frac{\mathbf{a} \cdot \mathbf{v}}{\left(1 - \frac{v^2}{c^2}\right)^{\hspace{-0.12em}{1/2}}} + \frac{(\mathbf{a} \cdot \mathbf{v}) \: (\mathbf{v} \cdot \mathbf{v})}{c^2\hspace{-0.15em}\left(1 - \frac{v^2}{c^2}\right)^{\hspace{-0.12em}{3/2}}} \; \right ] ~=~ \mathbf{F}\mathrm{\scriptscriptstyle{E}} \cdot \mathbf{v}
\end{eqnarray}
\par \vspace{+0.60em}
\begin{eqnarray}
m_o \left [ \; \frac{(\mathbf{a} \cdot \mathbf{v}) \: \mathbf{v}}{c^2\hspace{-0.15em}\left(1 - \frac{v^2}{c^2}\right)^{\hspace{-0.12em}{1/2}}} + \frac{(\mathbf{a} \cdot \mathbf{v}) \: (\mathbf{v} \cdot \mathbf{v}) \: \mathbf{v}}{c^{\hspace{+0.045em}4}\hspace{-0.15em}\left(1 - \frac{v^2}{c^2}\right)^{\hspace{-0.12em}{3/2}}} \; \right ] ~=~ \frac{(\mathbf{F}\mathrm{\scriptscriptstyle{E}} \cdot \mathbf{v}) \: \mathbf{v}}{c^2}
\end{eqnarray}
\par \vspace{+0.60em}
\begin{eqnarray}
m_o \left [ \; \frac{\mathbf{a}}{\left(1 - \frac{v^2}{c^2}\right)^{\hspace{-0.12em}{1/2}}} \; \right ] ~=~ \mathbf{F}\mathrm{\scriptscriptstyle{E}} - \frac{(\mathbf{F}\mathrm{\scriptscriptstyle{E}} \cdot \mathbf{v}) \: \mathbf{v}}{c^2}
\end{eqnarray}
\par \vspace{+0.60em}
\begin{eqnarray}
m_o \left [ \; \frac{\mathbf{a}}{\left(1 - \frac{v^2}{c^2}\right)^{\hspace{-0.12em}{1/2}}} \; \right ] ~=~ \mathbf{1} \cdot \mathbf{F}\mathrm{\scriptscriptstyle{E}} - \frac{(\mathbf{v} \otimes \mathbf{v}) \cdot \mathbf{F}\mathrm{\scriptscriptstyle{E}}}{c^2}
\end{eqnarray}
\par \vspace{+0.60em}
\begin{eqnarray}
m_o \left [ \; \frac{\mathbf{a}}{\left(1 - \frac{v^2}{c^2}\right)^{\hspace{-0.12em}{1/2}}} \; \right ] ~=~ \left [ \; \mathbf{1} - \frac{(\mathbf{v} \otimes \mathbf{v})}{c^2} \; \right ] \cdot \mathbf{F}\mathrm{\scriptscriptstyle{E}}
\end{eqnarray}
\par \vspace{+0.60em}
\begin{eqnarray}
m_o \left [ \; \frac{\mathbf{a}}{\left(1 - \frac{v^2}{c^2}\right)^{\hspace{-0.12em}{1/2}}} \; \right ] ~=~ \mathbf{M} \cdot \mathbf{F}\mathrm{\scriptscriptstyle{E}}
\end{eqnarray}
\par \vspace{+0.60em}
\begin{eqnarray}
m_o \left [ \; \frac{\mathbf{a}}{\left(1 - \frac{v^2}{c^2}\right)^{\hspace{-0.12em}{1/2}}} \; \right ] ~=~ \mathbf{F}\mathrm{\scriptscriptstyle{M}}
\end{eqnarray}
\par \vspace{+1.50em}
\noindent Note : $\mathbf{F}\mathrm{\scriptscriptstyle{E}} = \mathbf{1} \cdot \mathbf{F}\mathrm{\scriptscriptstyle{E}}$ (\hspace{+0.09em}$\mathbf{1}$ unit tensor\hspace{+0.09em}) {\footnotesize \&} $( \mathbf{F}\mathrm{\scriptscriptstyle{E}} \cdot \mathbf{v} ) \: \mathbf{v} = ( \mathbf{v} \otimes \mathbf{v} ) \cdot \mathbf{F}\mathrm{\scriptscriptstyle{E}}$ \hbox {(\hspace{+0.09em}$\otimes$ tensor or dyadic product\hspace{+0.09em})}

\newpage

\par \bigskip {\centering\subsection*{Annex II}}

\par \bigskip {\centering\subsection*{The Kinetic Forces}}\addcontentsline{toc}{subsection}{Annex II : The Kinetic Forces}

\bigskip \smallskip

\noindent The kinetic force \hbox {$\mathbf{K}^{a}_{\hspace{+0.012em}ij}$ exerted} on a particle $i$ with intrinsic mass $m_i$ by another particle $j$ with intrinsic mass $m_j$, \hbox {is given by}:
\par \vspace{-0.54em}
\begin{eqnarray}
\mathbf{K}^{a}_{\hspace{+0.012em}ij} \,=\, - \; \Bigg [ \; \dfrac{m_i \, m_j}{\mathbb{M}} \, ( \, \bar{\mathbf{a}}_{\hspace{+0.045em}i} \hspace{+0.045em}-\, \bar{\mathbf{a}}_{j} \, ) \; \Bigg ]
\end{eqnarray}
\par \vspace{+0.60em}
\noindent where $\bar{\mathbf{a}}_{\hspace{+0.045em}i}$ is the special acceleration of particle $i$, $\bar{\mathbf{a}}_{j}$ is the special acceleration of particle $j$ and $\mathbb{M}$ {\small ( $ = \sum_z^{\scriptscriptstyle{\mathit{All}}} m_z$ )} is the sum of the intrinsic masses of all the particles of the Universe.
\par \vspace{+0.60em}
\noindent On the other hand, the kinetic force $\mathbf{K}^{u}_{\hspace{+0.030em}i}$ exerted on a particle $i$ with intrinsic mass $m_i$ by the Universe, is given by:
\par \vspace{-0.45em}
\begin{eqnarray}
\mathbf{K}^{u}_{\hspace{+0.030em}i} \,=\, - \; m_i \; \dfrac{\sum_z^{\scriptscriptstyle{\mathit{All}}} m_z \, \bar{\mathbf{a}}_{\hspace{+0.045em}z}}{\sum_z^{\scriptscriptstyle{\mathit{All}}} m_z}
\end{eqnarray}
\par \vspace{+0.60em}
\noindent where $m_z$ and $\bar{\mathbf{a}}_{\hspace{+0.045em}z}$ are the intrinsic mass and the special acceleration of the \textit{z}-th particle of the Universe.
\par \vspace{+0.60em}
\noindent From the above equations it follows that the net kinetic force $\mathbf{K}_i$ {\small ( $ = \sum_j^{\scriptscriptstyle{\mathit{All}}} \, \mathbf{K}^{a}_{\hspace{+0.012em}ij}$} {\small $+ \; \mathbf{K}^{u}_{\hspace{+0.030em}i}$ )} acting on a particle $i$ with intrinsic mass $m_i$, is given by:
\par \vspace{-0.60em}
\begin{eqnarray}
\mathbf{K}_i \,=\, - \, m_i \, \bar{\mathbf{a}}_{\hspace{+0.045em}i}
\end{eqnarray}
\par \vspace{+0.60em}
\noindent where $\bar{\mathbf{a}}_{\hspace{+0.045em}i}$ is the special acceleration of particle $i$.
\par \vspace{+0.60em}
\noindent Now, from all dynamics $[ \, \hyperlink{efe}{(10)}, \hyperlink{efn}{(18)}, \hyperlink{efp}{(26)}, \hyperlink{efm}{(34)} \, ]$ we have:
\par \vspace{-0.60em}
\begin{eqnarray}
\mathbf{F}_i \,=\, m_i \, \bar{\mathbf{a}}_{\hspace{+0.045em}i}
\end{eqnarray}
\par \vspace{+0.60em}
\noindent Since {\small (\hspace{+0.240em}$\mathbf{K}_i \,=\, - \, m_i \, \bar{\mathbf{a}}_{\hspace{+0.045em}i}$\hspace{+0.240em})} we obtain:
\par \vspace{-0.81em}
\begin{eqnarray}
\mathbf{F}_i \,=\, - \, \mathbf{K}_i
\end{eqnarray}
\par \vspace{+0.30em}
\noindent that is:
\par \vspace{-0.81em}
\begin{eqnarray}
\mathbf{K}_i \, + \, \mathbf{F}_i \,=\, 0
\end{eqnarray}
\par \vspace{+0.60em}
\noindent If {\small (\hspace{+0.240em}$\mathbf{T}_i \,\doteq\, \mathbf{K}_i \, + \, \mathbf{F}_i$\hspace{+0.240em})} then:
\par \vspace{-0.81em}
\begin{eqnarray}
\mathbf{T}_i \,=\, 0
\end{eqnarray}
\par \vspace{+0.60em}
\noindent Therefore, if the net kinetic force $\mathbf{K}_i$ is added in all dynamics then the total \hbox {force $\mathbf{T}_i$} acting on a ( massive or non-massive ) particle $i$ is always zero.
\par \vspace{+0.21em}
\noindent Note : According to this paper, the kinetic forces ${\stackrel{\scriptstyle{au}}{\smash{\mathbf{K}}\rule{0pt}{+0.63em}}}$ are directly related to kinetic energy $\mathrm{K}$.

\newpage

\par \bigskip {\centering\subsection*{Annex III}}

\par \bigskip {\centering\subsection*{System of Particles}}\addcontentsline{toc}{subsection}{Annex III : System of Particles}

\bigskip \smallskip

\noindent In special relativity, the total energy $( \, \mathrm{E} \, )$ the linear momentum $( \, \mathbf{P} \, )$ the rest mass $( \, \mathrm{M}_o \, )$ and the velocity $( \, \mathbf{V} \, )$ of any massive or non-massive system ( of particles ) are given by:
\par \vspace{-0.30em}
\begin{eqnarray}
\mathrm{E} \hspace{-0.06em}~\doteq~ \sum \, m_i \, f_i \, c^2 + \sum \, \mathrm{E_{nki}}
\end{eqnarray}
\vspace{-0.60em}
\begin{eqnarray}
\mathbf{P} \hspace{-0.06em}~\doteq~ \sum \, m_i \, f_i \, \mathbf{v}_i
\end{eqnarray}
\vspace{-0.60em}
\begin{eqnarray}
\mathrm{M}_o^2 \, c^4 \hspace{-0.06em}~\doteq~ \mathrm{E}^2 - \mathbf{P}^2 c^2
\end{eqnarray}
\vspace{-0.60em}
\begin{eqnarray}
\mathbf{V} \hspace{-0.06em}~\doteq~ \mathbf{P} \: c^2 \, \mathrm{E}^{\scriptscriptstyle -1}
\end{eqnarray}
\par \vspace{+0.60em}
\noindent where $( \, m_i, \: f_i, \: \mathbf{v}_i \, )$ are the intrinsic mass, the relativistic factor and the velocity of the \hbox {\textit{i}-th} massive or non-massive particle of the system, $( \, \sum \, \mathrm{E_{nki}} \, )$ is the total non-kinetic energy of the system, \hbox {and $( \, c \, )$ is the speed} of light in vacuum.
\par \vspace{+0.60em}
\noindent The intrinsic mass $(\hspace{+0.06em}{\mathrm{M}}\hspace{+0.06em})$ and the relativistic factor $(\hspace{+0.06em}{\mathrm{F}}\hspace{+0.06em})$ of a massive system ( composed of massive particles or non-massive particles, or both at the same time ) \hbox {are given by}:
\par \vspace{-0.60em}
\begin{eqnarray}
\mathrm{M} ~\doteq~ \mathrm{M}_o
\end{eqnarray}
\vspace{-0.90em}
\begin{eqnarray}
\mathrm{F} ~\doteq~ \Big ( 1 - \dfrac{\mathbf{V} \cdot \mathbf{V}}{c^2} \hspace{+0.15em} \Big )^{\hspace{-0.24em}-\hspace{+0.03em}1/2}
\end{eqnarray}
\par \vspace{+0.60em}
\noindent where $( \, \mathrm{M}_o \, )$ is the rest mass of the massive system, $( \, \mathbf{V} \, )$ is the velocity of the massive system, and $( \, c \, )$ is the speed of light in vacuum.
\par \vspace{+0.60em}
\noindent The intrinsic mass $(\hspace{+0.06em}{\mathrm{M}}\hspace{+0.06em})$ and the relativistic factor $(\hspace{+0.06em}{\mathrm{F}}\hspace{+0.06em})$ of a non-massive system ( composed only of non-massive particles, all with the same vector velocity $\mathbf{c}$ ) \hbox {are given by}:
\par \vspace{-0.60em}
\begin{eqnarray}
\mathrm{M} ~\doteq~ \dfrac{h \, \kappa}{c^2}
\end{eqnarray}
\vspace{-0.60em}
\begin{eqnarray}
\mathrm{F} ~\doteq~ \frac{1}{\kappa} \, \sum \, \nu_i
\end{eqnarray}
\par \vspace{+0.60em}
\noindent where $( \hspace{+0.33em} h \hspace{+0.33em} )$ is the Planck constant, \hspace{+0.06em}$( \hspace{+0.30em} \nu_i \hspace{+0.30em} )$ is the frequency of the \textit{i}-th non-massive particle of the non-massive system, $( \, \kappa \, )$ is a positive universal constant with dimension of frequency\hspace{-0.03em}, and $( \, c \, )$ is the speed of light in vacuum.
\par \vspace{+0.60em}
\noindent According to this paper, a massive system $( \, \mathrm{M}_o \ne 0 \, )$ is a system with non-zero rest mass \hbox {( or a system} whose speed $\mathrm{V}$ in vacuum is less than $c$ ) and a non-massive system $( \, \mathrm{M}_o = 0 \, )$ is a system with zero rest mass ( or a system whose speed $\mathrm{V}$ in vacuum is $c$ )
\par \vspace{+0.60em}
\noindent Note : The rest mass $( \, \mathrm{M}_o \, )$ and the intrinsic mass $( \, \mathrm{M} \, )$ are in general not additive, and the relativistic mass $( \, {\mathtt{M}} \, )$ of a system ( massive or non-massive ) is given by : \hbox {\hspace{-0.09em}( ${\mathtt{M}} ~\doteq~ \mathrm{M} \, \mathrm{F}$ )}

\newpage

\par \bigskip {\centering\subsection*{The Einsteinian Kinematics}}

\bigskip \smallskip

\noindent The special position $( \, \bar{\mathbf{\hspace{+0.12em}R}} \: )$ the special velocity $( \, \bar{\mathbf{V}} \, )$ and the special acceleration $( \, \bar{\mathbf{A}} \, )$ of a system ( massive or non-massive ) are given by:
\par \vspace{-0.30em}
\begin{eqnarray}
\bar{\mathbf{\hspace{+0.12em}R}}\hspace{-0.24em} \hspace{+0.12em}~\doteq~ \int \mathrm{F} \, \mathbf{V} \; d\hspace{+0.012em}t
\end{eqnarray}
\vspace{-0.45em}
\begin{eqnarray}
\bar{\mathbf{V}} \hspace{-0.03em}~\doteq~ \dfrac{d\hspace{-0.09em}\bar{\mathbf{\hspace{+0.12em}R}}}{d\hspace{+0.012em}t} ~=~ \mathrm{F} \, \mathbf{V}
\end{eqnarray}
\vspace{-0.30em}
\begin{eqnarray}
\bar{\mathbf{A}} \hspace{+0.03em}~\doteq~ \dfrac{d\hspace{+0.03em}\bar{\mathbf{V}}}{d\hspace{+0.012em}t} ~=~ \mathrm{F} \, \dfrac{d\hspace{+0.03em}\mathbf{V}}{d\hspace{+0.012em}t} + \dfrac{d\hspace{+0.03em}\mathrm{F}}{d\hspace{+0.012em}t} \, \mathbf{V}
\end{eqnarray}
\par \vspace{+1.20em}
\noindent where $( \, \mathrm{F} \, )$ is the relativistic factor of the system, $( \, \mathbf{V} \, )$ is the velocity of the system, and $( \, t \, )$ is the (\hspace{+0.15em}coordinate\hspace{+0.15em}) time.

\vspace{+0.60em}

\par \bigskip {\centering\subsection*{The Einsteinian Dynamics}}

\bigskip \smallskip

\noindent If we consider a system ( massive or non-massive ) with intrinsic mass $( \, \mathrm{M} \, )$ then the \hbox {linear} momentum $( \, \mathbf{P} \, )$ of the system, the angular momentum $( \, \mathbf{L} \, )$ of the system, the net \hbox {Einsteinian} force $( \, \mathbf{F} \, )$ acting on the system, the work $( \, \mathrm{W} \, )$ done by the net Einsteinian forces acting on the system, the kinetic energy $( \, \mathrm{K} \, )$ of the system, and the total energy $( \, \mathrm{E} \, )$ \hbox {of the system, are:}
\par \vspace{-0.30em}
\begin{eqnarray}
\mathbf{P} \hspace{-0.06em}~\doteq~ \sum \, \mathbf{p}_i ~=~ \sum \, m_i \, \bar{\mathbf{v}}_i ~=~ \sum \, m_i \, f_i \, \mathbf{v}_i ~=~ \mathrm{M} \, \bar{\mathbf{V}} ~=~ \mathrm{M} \, \mathrm{F} \, \mathbf{V}
\end{eqnarray}
\vspace{-0.30em}
\begin{eqnarray}
\mathbf{L} \hspace{+0.03em}~\doteq~ \sum \, \mathbf{l}_{\hspace{+0.06em}i} ~=~ \sum \, \mathbf{r}_i \times \mathbf{p}_i ~=~ \sum \, m_i \; \mathbf{r}_i \times \bar{\mathbf{v}}_i ~=~ \sum \, m_i \, f_i \: \mathbf{r}_i \times \mathbf{v}_i
\end{eqnarray}
\vspace{-0.30em}
\begin{eqnarray}
\mathbf{F} ~=~ \sum \, \mathbf{f}_{\hspace{+0.06em}i} ~=~ \sum \, \dfrac{d\hspace{+0.045em}\mathbf{p}_i}{d\hspace{+0.012em}t} ~=~ \dfrac{d\hspace{+0.045em}\mathbf{P}}{d\hspace{+0.012em}t} ~=~ \mathrm{M} \, \bar{\mathbf{A}} ~=~ \mathrm{M} \, \bigg [ \, \mathrm{F} \, \dfrac{d\hspace{+0.03em}\mathbf{V}}{d\hspace{+0.012em}t} + \dfrac{d\hspace{+0.03em}\mathrm{F}}{d\hspace{+0.012em}t} \, \mathbf{V} \, \bigg ]
\end{eqnarray}
\vspace{-0.15em}
\begin{eqnarray}
\mathrm{W} \hspace{-0.312em}~\doteq~ \sum \, \int_{\scriptscriptstyle 1}^{\hspace{+0.09em}{\scriptscriptstyle 2}} \mathbf{f}_{\hspace{+0.06em}i} \cdot d\hspace{+0.036em}\mathbf{r}_i ~=~ \sum \, \int_{\scriptscriptstyle 1}^{\hspace{+0.09em}{\scriptscriptstyle 2}} \dfrac{d\hspace{+0.045em}\mathbf{p}_i}{d\hspace{+0.012em}t} \cdot d\hspace{+0.036em}\mathbf{r}_i ~=~ \Delta \, \mathrm{K}
\end{eqnarray}
\vspace{-0.30em}
\begin{eqnarray}
\mathrm{K} \hspace{-0.06em}~\doteq~ \sum \, m_i \, f_i \, c^2
\end{eqnarray}
\vspace{-0.30em}
\begin{eqnarray}
\mathrm{E} \hspace{-0.06em}~\doteq~ \sum \, m_i \, f_i \, c^2 + \sum \, \mathrm{E_{nki}} ~=~ \mathrm{K} + \sum \, \mathrm{E_{nki}} ~=~ \mathrm{M} \, \mathrm{F} \, c^2
\end{eqnarray}
\par \vspace{+0.90em}
\noindent where $( \: m_i, \: f_i, \: \mathbf{r}_i, \: \mathbf{v}_i, \: \bar{\mathbf{v}}_i \: )$ are the intrinsic mass, the relativistic factor, the position, the velocity and the special velocity of the \textit{i}-th massive or non-massive particle of the system, $( \: \mathrm{F}, \: \mathbf{V}, \: \bar{\mathbf{V}}, \: \bar{\mathbf{A}} \: )$ are the relativistic factor, the velocity, the special velocity and the special acceleration of the system, $( \, \sum \, \mathrm{E_{nki}} \, )$ is the total non-kinetic energy of the system, $( \, t \, )$ is the \hbox {(\hspace{+0.15em}coordinate\hspace{+0.15em})} time, and $( \, c \, )$ is the speed of light in vacuum.
\par \vspace{+0.60em}
\noindent {\small Note :} {\footnotesize (\hspace{+0.24em}$\sum \, \mathrm{E_{nki}} \,=\, 0$\hspace{+0.24em}) in massive or non-massive particle \hspace{+0.36em}$\rightarrow$\hspace{+0.36em} (\hspace{+0.24em}$\mathrm{E} \,=\, \mathrm{K}$\hspace{+0.24em}) in massive or non-massive particle.}

\newpage

\par \bigskip {\centering\subsection*{Appendix A}}\addcontentsline{toc}{subsection}{Appendix A : The Minkowskian Four-Force}\hypertarget{apa}{}

\vspace{+0.00em}

\par \bigskip {\centering\subsection*{Four-kinematics}}

\vspace{+0.00em}

\par \bigskip {\centering\subsection*{The Minkowskian Kinematics}}

\bigskip \smallskip

\noindent The special four-position $( \, \mathsf{R} \, )$ the special four-velocity $( \, \mathsf{U} \, )$ and the special four-acceleration $( \, \mathsf{A} \, )$ of a particle ( massive or non-massive ) are given by:
\par \vspace{-0.30em}
\begin{eqnarray}
\mathsf{R} \hspace{+0.00em}~\doteq~ \bigg ( \, c \, t \;\, , \; \mathbf{r} \, \bigg )
\end{eqnarray}
\vspace{-0.45em}
\begin{eqnarray}
\mathsf{U} \hspace{-0.03em}~\doteq~ \dfrac{d\hspace{+0.036em}\mathsf{R}}{d\hspace{+0.012em}\tau} ~=~ \bigg ( \, f \, c \;\, , \; f \, \mathbf{v} \, \bigg )
\end{eqnarray}
\vspace{-0.30em}
\begin{eqnarray}
\mathsf{A} \hspace{-0.03em}~\doteq~ \dfrac{d\hspace{+0.036em}\mathsf{U}}{d\hspace{+0.012em}\tau} ~=~ f \: \bigg ( \, \dfrac{d\hspace{-0.12em}f}{d\hspace{+0.012em}t} \, c \;\, , \; \dfrac{d\hspace{-0.12em}f}{d\hspace{+0.012em}t} \, \mathbf{v} + \dfrac{d\hspace{+0.021em}\mathbf{v}}{d\hspace{+0.012em}t} \, f \, \bigg )
\end{eqnarray}
\par \vspace{+1.20em}
\noindent where $( \, f \, )$ is the relativistic factor of the particle, $( \, \mathbf{r} \, )$ is the position of the particle, $( \, \mathbf{v} \, )$ is the velocity of the particle, and $( \, \tau \, )$ is the proper time of the particle ( Note : $d\tau \,=\, f^{\scriptscriptstyle -1} \: dt$ )

\vspace{+0.60em}

\par \bigskip {\centering\subsection*{Four-dynamics}}

\vspace{+0.00em}

\par \bigskip {\centering\subsection*{The Minkowskian Dynamics}}

\bigskip \smallskip

\noindent The four-momentum $( \, \mathbf{\overline{P}} \, )$ of a particle ( massive or non-massive ) with intrinsic mass $( \, m \, )$ and the net Minkowskian four-force $( \, \mathbf{\overline{F}}\mathrm{\scriptscriptstyle{M}} \, )$ acting on the particle, are given by:
\par \vspace{-0.30em}
\begin{eqnarray}
\mathbf{\overline{P}} \hspace{-0.06em}~\doteq~ m \, \mathsf{U} ~=~ m \: \bigg ( \, f \, c \;\, , \; f \, \mathbf{v} \, \bigg )
\end{eqnarray}
\vspace{-0.30em}
\begin{eqnarray}
\mathbf{\overline{F}}\mathrm{\scriptscriptstyle{M}} ~=~ \dfrac{d\hspace{+0.045em}\mathbf{\overline{P}}}{d\hspace{+0.012em}\tau} ~=~ m \, \mathsf{A} ~=~ m \: f \: \bigg ( \, \dfrac{d\hspace{-0.12em}f}{d\hspace{+0.012em}t} \, c \;\, , \; \dfrac{d\hspace{-0.12em}f}{d\hspace{+0.012em}t} \, \mathbf{v} + \dfrac{d\hspace{+0.021em}\mathbf{v}}{d\hspace{+0.012em}t} \, f \, \bigg )
\end{eqnarray}
\par \vspace{+1.20em}
\noindent where $( \: f, \: \mathbf{v}, \: \mathsf{U}, \: \mathsf{A} \: )$ are the relativistic factor, the velocity, the special four-velocity and the special four-acceleration of the particle, $( \, \tau \, )$ is the proper time of the particle, and $( \, c \, )$ is the speed of light in vacuum.
\par \vspace{+0.90em}
\noindent In the Minkowskian four-mechanics ( that is, in the ordinary four-mechanics ) all the special four-vectors $( \, \mathsf{R}, \: \mathsf{U}, \: \mathsf{A}, \: \mathbf{\overline{P}}, \: \mathbf{\overline{F}}\mathrm{\scriptscriptstyle{M}} \, )$ are ordinary four-vectors $( \, \mathbf{R}, \: \mathbf{U}, \: \mathbf{A}, \: \mathbf{P}, \: \mathbf{F} \, )$\hspace{+0.03em}.
\par \vspace{+0.90em}
\noindent Additionally, in massive particle : $f$ is the Lorentz factor $\gamma(\mathbf{v})$\hspace{+0.09em}.

\newpage

\par \bigskip {\centering\subsection*{Appendix B}}\addcontentsline{toc}{subsection}{Appendix B : The Einsteinian Four-Force}\hypertarget{apb}{}

\vspace{+0.00em}

\par \bigskip {\centering\subsection*{Four-kinematics}}

\vspace{+0.00em}

\par \bigskip {\centering\subsection*{The Einsteinian Kinematics}}

\bigskip \smallskip

\noindent The special four-position $( \, \mathsf{R} \, )$ the special four-velocity $( \, \mathsf{U} \, )$ and the special four-acceleration $( \, \mathsf{A} \, )$ of a particle ( massive or non-massive ) are given by:
\par \vspace{-0.30em}
\begin{eqnarray}
\mathsf{R} \hspace{+0.00em}~\doteq~ \int \bigg ( \, f \, c \;\, , \; f \, \mathbf{v} \, \bigg ) \; d\hspace{+0.012em}t
\end{eqnarray}
\vspace{-0.45em}
\begin{eqnarray}
\mathsf{U} \hspace{-0.03em}~\doteq~ \dfrac{d\hspace{+0.036em}\mathsf{R}}{d\hspace{+0.012em}t} ~=~ \bigg ( \, f \, c \;\, , \; f \, \mathbf{v} \, \bigg )
\end{eqnarray}
\vspace{-0.30em}
\begin{eqnarray}
\mathsf{A} \hspace{-0.03em}~\doteq~ \dfrac{d\hspace{+0.036em}\mathsf{U}}{d\hspace{+0.012em}t} ~=~ \bigg ( \, \dfrac{d\hspace{-0.12em}f}{d\hspace{+0.012em}t} \, c \;\, , \; \dfrac{d\hspace{-0.12em}f}{d\hspace{+0.012em}t} \, \mathbf{v} + \dfrac{d\hspace{+0.021em}\mathbf{v}}{d\hspace{+0.012em}t} \, f \, \bigg )
\end{eqnarray}
\par \vspace{+1.20em}
\noindent where $( \, f \, )$ is the relativistic factor of the particle, $( \, \mathbf{v} \, )$ is the velocity of the particle, and $( \, t \, )$ is the (\hspace{+0.15em}coordinate\hspace{+0.15em}) time.

\vspace{+0.60em}

\par \bigskip {\centering\subsection*{Four-dynamics}}

\vspace{+0.00em}

\par \bigskip {\centering\subsection*{The Einsteinian Dynamics}}

\bigskip \smallskip

\noindent The four-momentum $( \, \mathbf{\overline{P}} \, )$ of a particle ( massive or non-massive ) with intrinsic mass $( \, m \, )$ and the net Einsteinian four-force $( \, \mathbf{\overline{F}}\mathrm{\scriptscriptstyle{E}} \, )$ acting on the particle, are given by:
\par \vspace{-0.30em}
\begin{eqnarray}
\mathbf{\overline{P}} \hspace{-0.06em}~\doteq~ m \, \mathsf{U} ~=~ m \: \bigg ( \, f \, c \;\, , \; f \, \mathbf{v} \, \bigg )
\end{eqnarray}
\vspace{-0.30em}
\begin{eqnarray}
\mathbf{\overline{F}}\mathrm{\scriptscriptstyle{E}} ~=~ \dfrac{d\hspace{+0.045em}\mathbf{\overline{P}}}{d\hspace{+0.012em}t} ~=~ m \: \mathsf{A} ~=~ m \, \bigg ( \, \dfrac{d\hspace{-0.12em}f}{d\hspace{+0.012em}t} \, c \;\, , \; \dfrac{d\hspace{-0.12em}f}{d\hspace{+0.012em}t} \, \mathbf{v} + \dfrac{d\hspace{+0.021em}\mathbf{v}}{d\hspace{+0.012em}t} \, f \, \bigg )
\end{eqnarray}
\par \vspace{+1.20em}
\noindent where $( \: f, \: \mathbf{v}, \: \mathsf{U}, \: \mathsf{A} \: )$ are the relativistic factor, the velocity, the special four-velocity and the special four-acceleration of the particle, $( \, t \, )$ is the (\hspace{+0.15em}coordinate\hspace{+0.15em}) time, and $( \, c \, )$ is the speed of light in vacuum.
\par \vspace{+0.90em}
\noindent In the Einsteinian four-mechanics, the special four-velocity $( \, \mathsf{U} \, )$ is the ordinary four-velocity $( \, \mathbf{U} \, )$ and, therefore, the four-momentum $( \, \mathbf{\overline{P}} \, )$ is the ordinary four-momentum $( \, \mathbf{P} \, )$\hspace{+0.03em}.
\par \vspace{+0.90em}
\noindent Additionally, in massive particle : $f$ is the Lorentz factor $\gamma(\mathbf{v})$\hspace{+0.09em}.

\end{document}

