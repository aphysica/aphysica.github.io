
\documentclass[10pt,fleqn]{article}
%\documentclass[a4paper,10pt]{article}�
%\documentclass[letterpaper,10pt]{article}

%\usepackage[dvips]{geometry}
%\geometry{papersize={162.0mm,234.0mm}}
%\geometry{totalwidth=141.0mm,totalheight=198.0mm}
\usepackage[papersize={162.0mm,258.0mm},totalwidth=141.0mm,totalheight=222.0mm]{geometry}

\usepackage[english]{babel}
\usepackage[latin1]{inputenc}
%\usepackage[utf8]{inputenc}
%\usepackage[T1]{fontenc}
%\usepackage{ae,aecompl}
%\usepackage{mathptmx}
%\usepackage{pslatex}
%\usepackage{lmodern}
\usepackage{amsfonts}
\usepackage{amsmath}
\usepackage{amssymb}

%\frenchspacing

\usepackage{hyperref}
\hypersetup{colorlinks=true,linkcolor=black,urlcolor=blue,bookmarksopen=true}
\hypersetup{bookmarksnumbered=true,pdfstartview=FitH,pdfpagemode=UseOutlines}
%\hypersetup{bookmarksnumbered=true,pdfstartview=FitH,pdfpagemode=UseNone}
\hypersetup{pdftitle={Special Relativity: Types of Energy}}
\hypersetup{pdfauthor={A. Torassa}}

\setlength{\arraycolsep}{1.74pt}

\newcommand{\vet}{\medskip}
\newcommand{\vek}{\bigskip \bigskip}

%\ccl Attribution 4.0 International License
%\cct Attribution 4.0 International License
%\newcommand{\ccl}{${\large{\textcircled{\raisebox{+0.8pt}{{\scriptsize{{\textsf{cc}}}}}}}}$\;}
%\newcommand{\cct}{${\normalsize{\textcircled{\raisebox{+1.00pt}{{\tiny{{\texttt{cc}}}}}}}}$\;}

\newcommand{\med}{\raise.5ex\hbox{$\scriptstyle 1$}\kern-.15em/\kern-.09em\lower.25ex\hbox{$\scriptstyle 2$}}
\newcommand{\met}{\raise.5ex\hbox{$\scriptstyle 1$}\kern-.15em/\kern-.09em\lower.25ex\hbox{$\scriptstyle 2$}}

\begin{document}

\enlargethispage{+0.00em}

\addcontentsline{toc}{section}{Special Relativity: Types of Energy}

\begin{center}

{\LARGE Special Relativity: Types of Energy}

\vek

{\large A. Torassa}

\vek

\small

Creative Commons Attribution 4.0 License

\vet

\href{https://orcid.org/0000-0002-1389-247X}{\scriptsize{ORCID}} \S \ (2025) Buenos Aires
%\href{https://orcid.org/0000-0002-1389-247X}{\normalsize{\textsc{orcid}}} \S \ (2025) Buenos Aires

\vet

Argentina

\bigskip \bigskip

\parbox{93.90mm}{In special relativity, this paper presents the definitions of kinetic energy, rest energy and relativistic energy for a single particle \hbox {( massive} or non-massive ) Additionally, this paper also presents the definitions of generalized relativistic energy and total energy for a system of particles ( massive and non-massive )}

\end{center}

\normalsize

\vspace{-1.50em}

\par \bigskip {\centering\subsection*{Introduction}}\addcontentsline{toc}{subsection}{1. Introduction}

\bigskip \smallskip

\noindent In special relativity, this paper is obtained starting from the essential definitions of intrinsic mass ( or invariant mass ) and relativistic factor ( or frequency factor ) for massive particles and non-massive particles.
\par \vspace{+0.60em}
\noindent The intrinsic mass $( \, m \, )$ and the relativistic factor $( \, f \, )$ of a massive particle, \hbox {are given by}:
\par \vspace{-0.60em}
\begin{eqnarray}
m ~\doteq~ m_o
\end{eqnarray}
\vspace{-0.90em}
\begin{eqnarray}
f ~\doteq~ \Big ( 1 - \dfrac{\mathbf{v} \cdot \mathbf{v}}{c^2} \hspace{+0.15em} \Big )^{\hspace{-0.24em}-\hspace{+0.03em}1/2}
\end{eqnarray}
\par \vspace{+0.60em}
\noindent where $( \, m_o \, )$ is the rest mass of the massive particle, $( \, \mathbf{v} \, )$ is the velocity of the massive particle, and $( \, c \, )$ is the speed of light in vacuum.
\par \vspace{+0.60em}
\noindent The intrinsic mass $( \, m \, )$ and the relativistic factor $( \, f \, )$ of a non-massive particle, \hbox {are given by}:
\par \vspace{-0.60em}
\begin{eqnarray}
m ~\doteq~ \dfrac{h \, \kappa}{c^2}
\end{eqnarray}
\vspace{-0.60em}
\begin{eqnarray}
f ~\doteq~ \dfrac{\nu}{\kappa}
\end{eqnarray}
\par \vspace{+0.60em}
\noindent where $( \hspace{+0.33em} h \hspace{+0.33em} )$ is the Planck constant, \hspace{+0.06em}$( \hspace{+0.30em} \nu \hspace{+0.30em} )$ is the frequency of the \hbox {non-massive} particle, $( \, \kappa \, )$ is a positive universal constant with dimension of frequency\hspace{-0.03em}, and $( \, c \, )$ is the speed of light in vacuum.
\par \vspace{+0.60em}
\noindent According to this paper, a massive particle $( \, m_o \ne 0 \, )$ is a particle with non-zero rest mass ( or alternatively is a particle whose speed $v$ in vacuum is less than $c$ ) and a non-massive particle $( \, m_o = 0 \, )$ is a particle with zero rest mass ( or alternatively is a particle whose speed $v$ in vacuum is $c$ )
\par \vspace{+0.60em}
\noindent Note : The rest mass $( \, m_o \, )$ and the intrinsic mass $( \, m \, )$ are in general not additive, and the relativistic mass $( \, {\mathrm{m}} \, )$ of a particle ( massive or non-massive ) is given by : \hbox {\hspace{-0.09em}( ${\mathrm{m}} ~\doteq~ m \, f$ )}

\newpage

\par \bigskip {\centering\subsection*{The Einsteinian Kinematics}}\addcontentsline{toc}{subsection}{2. The Einsteinian Kinematics}

\bigskip \smallskip

\noindent The special position $( \, \bar{\mathbf{r}} \, )$ the special velocity $( \, \bar{\mathbf{v}} \, )$ and the special acceleration $( \, \bar{\mathbf{a}} \, )$ of a particle \hbox {( massive or non-massive )} are given by:
\par \vspace{-0.30em}
\begin{eqnarray}
\bar{\mathbf{r}} \hspace{+0.12em}~\doteq~ \int f \, \mathbf{v} \; d\hspace{+0.012em}t
\end{eqnarray}
\vspace{-0.45em}
\begin{eqnarray}
\bar{\mathbf{v}} \hspace{-0.03em}~\doteq~ \dfrac{d\hspace{+0.036em}\bar{\mathbf{r}}}{d\hspace{+0.012em}t} ~=~ f \, \mathbf{v}
\end{eqnarray}
\vspace{-0.30em}
\begin{eqnarray}
\bar{\mathbf{a}} \hspace{+0.03em}~\doteq~ \dfrac{d\hspace{+0.021em}\bar{\mathbf{v}}}{d\hspace{+0.012em}t} ~=~ f \, \dfrac{d\hspace{+0.021em}\mathbf{v}}{d\hspace{+0.012em}t} + \dfrac{d\hspace{-0.12em}f}{d\hspace{+0.012em}t} \, \mathbf{v}
\end{eqnarray}
\par \vspace{+1.20em}
\noindent where $( \, f \, )$ is the relativistic factor of the particle, $( \, \mathbf{v} \, )$ is the velocity of the particle, and $( \, t \, )$ is the (\hspace{+0.15em}coordinate\hspace{+0.15em}) time.

\vspace{+0.60em}

\par \bigskip {\centering\subsection*{The Einsteinian Dynamics}}\addcontentsline{toc}{subsection}{3. The Einsteinian Dynamics}

\bigskip \smallskip

\noindent If we consider a particle ( massive or non-massive ) with intrinsic mass $( \, m \, )$ then the \hbox {linear} momentum $( \, \mathbf{P} \, )$ of the particle, the angular momentum $( \, \mathbf{L} \, )$ of the particle, the net \hbox {Einsteinian} force $( \, \mathbf{F}\mathrm{\scriptscriptstyle{E}} \, )$ acting on the particle, the work $( \, \mathrm{W} \, )$ done by the net Einsteinian force acting on the particle, the kinetic energy $( \, \mathrm{K} \, )$ of the particle, the rest energy $( \, \mathrm{E}_o \, )$ of the particle, and the relativistic energy $( \, \mathrm{E} \, )$ of the particle, they are given by:
\par \vspace{-0.30em}
\begin{eqnarray}
\mathbf{P} \hspace{-0.06em}~\doteq~ m \, \bar{\mathbf{v}} ~=~ m \, f \, \mathbf{v}
\end{eqnarray}
\vspace{-0.30em}
\begin{eqnarray}
\mathbf{L} \hspace{+0.03em}~\doteq~ \mathbf{r} \times \mathbf{P} ~=~ m \; \mathbf{r} \times \bar{\mathbf{v}} ~=~ m \, f \: \mathbf{r} \times \mathbf{v}
\end{eqnarray}
\vspace{-0.30em}
\hypertarget{efe}{}
\begin{eqnarray}
\mathbf{F}\mathrm{\scriptscriptstyle{E}} ~=~ \dfrac{d\hspace{+0.045em}\mathbf{P}}{d\hspace{+0.012em}t} ~=~ m \, \bar{\mathbf{a}} ~=~ m \, \bigg [ \, f \, \dfrac{d\hspace{+0.021em}\mathbf{v}}{d\hspace{+0.012em}t} + \dfrac{d\hspace{-0.12em}f}{d\hspace{+0.012em}t} \, \mathbf{v} \, \bigg ]
\end{eqnarray}
\vspace{-0.15em}
\begin{eqnarray}
\mathrm{W} \hspace{-0.312em}~\doteq~ \hspace{-0.36em} \int_{\scriptscriptstyle 1}^{\hspace{+0.09em}{\scriptscriptstyle 2}} \mathbf{F}\mathrm{\scriptscriptstyle{E}} \cdot d\hspace{+0.036em}\mathbf{r} ~=~ \Delta \, \mathrm{K} ~=~ \Delta \, \mathrm{E}
\end{eqnarray}
\vspace{-0.30em}
\begin{eqnarray}
\mathrm{K} \hspace{-0.06em}~\doteq~ m \, f \, c^2 - m_o \, c^2
\end{eqnarray}
\vspace{-0.30em}
\begin{eqnarray}
\mathrm{E}_o \hspace{-0.06em}~\doteq~ m_o \, c^2
\end{eqnarray}
\vspace{-0.30em}
\begin{eqnarray}
\mathrm{E} \hspace{-0.06em}~\doteq~ \mathrm{K} + \mathrm{E}_o \,=\, m \, f \, c^2
\end{eqnarray}
\par \vspace{+0.90em}
\noindent where $( \: f, \: \mathbf{r}, \: \mathbf{v}, \: \bar{\mathbf{v}}, \: \bar{\mathbf{a}} \: )$ are the relativistic factor, the position, the velocity, the special velocity and the special acceleration of the particle, $( \, t \, )$ is the (\hspace{+0.15em}coordinate\hspace{+0.15em}) time, and $( \, c \, )$ is the speed of light in vacuum. The kinetic energy $( \, \mathrm{K}_o \, )$ of a massive particle at rest is always zero since in this dynamics the relativistic energy $( \, \mathrm{E} \, )$ and the kinetic energy $( \, \mathrm{K} \, )$ of a massive particle are not the same $( \, \mathrm{E} \ne \mathrm{K} \, )$ [ Note : in non-massive particle : $m_o \,=\, 0$ , therefore : $\mathrm{E}_o \,=\, 0$ ]
\par \vspace{+0.60em}
\noindent {\small Note :} {\footnotesize $\mathrm{E}^2 - \mathbf{P}^2 c^2 = m^2 \, f^2 \, c^4 \, (\hspace{+0.12em} 1 - \mathbf{v}^2/c^2 \hspace{+0.15em})$} {\small [ in massive particle :} {\footnotesize $f^2 \, (\hspace{+0.12em} 1 - \mathbf{v}^2/c^2 \hspace{+0.15em}) = 1$ \hspace{+0.27em}$\rightarrow$\hspace{+0.27em} \hbox {{$\mathrm{E}^2 - \mathbf{P}^2 c^2 = {m_o}^2 c^4$}} \,and\, $m \ne 0$} {\small ]} \,{\footnotesize\&}\, {\small [ in non-massive particle :} {\footnotesize $\mathbf{v}^2 = c^2$ \hspace{+0.27em}$\rightarrow$\hspace{+0.27em} $(\hspace{+0.12em} 1 - \mathbf{v}^2/c^2 \hspace{+0.15em}) = 0$ \hspace{+0.27em}$\rightarrow$\hspace{+0.27em} \hbox {{$\mathrm{E}^2 - \mathbf{P}^2 c^2 = 0$ \:and\: $m \ne 0$} {\small ]}}}

\newpage

\par \bigskip {\centering\subsection*{General Observations}}\addcontentsline{toc}{subsection}{4. General Observations}

\bigskip \smallskip

\noindent In classical mechanics, the kinetic energy $( \, \mathrm{K} \, )$ of a massive particle $( \, m_o \, )$ is given by the following indefinite integral : $\mathrm{K} ~= \int m_o \, \mathbf{a} \cdot d\hspace{+0.036em}\mathbf{r} ~=~ \med \; m_o \, v^2 + \mathrm{constant}$
\par \smallskip
\noindent In special relativity, the kinetic energy $( \, \mathrm{K} \, )$ of a massive particle $( \, m_o \, )$ is given by the following indefinite integral : $\mathrm{K} ~= \int m_o \, \bar{\mathbf{a}} \cdot d\hspace{+0.036em}\mathbf{r} ~=~ m_o \, f \, c^2 + \mathrm{constant}$
\par \smallskip
\noindent Basically, there are two criteria for choosing a value for the constant of integration of the kinetic energy. The first criterion states that the constant of integration must be such that the kinetic energy of any massive particle at rest must always be zero and the second criterion simply states that the constant of integration must always be zero.
\par \smallskip
\noindent In classical mechanics, both criteria lead to the same result : $( \, \mathrm{K} \hspace{-0.06em}~\doteq~ \med \; m_o \, v^2 \, )$ However, in special relativity, the first and second criteria lead to different results, with the first \hbox {criterion :} $( \, \mathrm{K} \hspace{-0.06em}~\doteq~ m_o \, f \, c^2 - m_o \, c^2 \, )$ and with the second criterion : $( \, \mathrm{K} \hspace{-0.06em}~\doteq~ m_o \, f \, c^2 \, )$
\par \smallskip
\noindent The first and second criteria are arbitrary since ( in classical mechanics and also in special relativity ) the kinetic energy of a particle depends on the speed of the particle and also on the mass of the particle. For example, a particle A can have more kinetic energy than another particle B whose speed is much higher than the speed of particle A.
\par \smallskip
\noindent Using the first criterion simply because this type of energy is identified with the adjective `kinetic' ( pertaining to motion ) is even more arbitrary, since this type of energy would be better identified if it were called, for example : mass-speed energy.
\par \smallskip
\noindent On the other hand, according to this paper, each type of energy is associated with a type of force. If we use the second integration criterion then the kinetic energy $( \, \mathrm{K} \, )$ would be \hbox {associated} with the net kinetic force $( \, \mathbf{K} \, )$ (\hspace{+0.15em}see \hyperlink{p3a2}{{\small A}nnex {\small II}}\hspace{+0.24em}) However, if we use the first \hbox {integration} criterion then the kinetic energy $( \, \mathrm{K} \, )$ would also be associated with the net \hbox {kinetic force} $( \, \mathbf{K} \, )$ but the rest energy $( \, \mathrm{E}_o \, )$ would be associated with another type of force that would act only on massive particles and whose value would always be zero.
\par \smallskip
\noindent Throughout this paper we will use the second integration criterion ( because it is the easiest to use from a theoretical point of view ) Therefore, in special relativity, the kinetic energy $( \, \mathrm{K}_o \, )$ of a massive particle at rest is $( \, m_o \, c^2 \, )$ since using the second integration criterion the kinetic energy $( \, \mathrm{K} \, )$ and the relativistic energy $( \, \mathrm{E} \, )$ of a single massive or non-massive particle are the same $( \, \mathrm{K} \hspace{-0.06em}~\doteq~ m \, f \, c^2 \,=\, \mathrm{E} \, )$ \{ \hyperlink{ref}{[\,1\,]} \hyperlink{ref}{[\,2\,]} \hyperlink{ref}{[\,3\,]} [\,\hyperlink{p3a1}{{\small A}nnex {\small I}}\;] [\,\hyperlink{p3a2}{{\small A}nnex {\small II}}\;] \}
\par \smallskip
\noindent Finally (\hspace{+0.15em}see \hyperlink{p3a1}{{\small A}nnex {\small I}}\hspace{+0.24em}) the generalized relativistic energy $( \, \mathrm{E} \, )$ and the total energy $( \, \mathrm{T} \, )$ of a system of particles ( massive and non-massive ) are defined. Here, for example, a system of particles composed only of a single massive or non-massive particle can have non-kinetic energy ( also called `potential' energy ) ( in this example : $\sum \, \mathrm{E_{nke}}$ )
\par \smallskip
\noindent On the other hand, in massive systems ( of particles ) there are apparently at least two types of ordinary velocities : {\small $( \, \mathbf{V}\mathrm{\scriptscriptstyle{E}} \hspace{-0.06em}~\doteq~ \mathbf{P} \: c^2 \, \mathrm{E}^{\scriptscriptstyle -1} \, )$ $[ \, \hyperlink{eve}{19} \, ]$} and {\small $( \, \mathbf{V}\mathrm{\scriptscriptstyle{K}} \hspace{-0.06em}~\doteq~ \mathbf{P} \: c^2 \, \mathrm{K}^{\scriptscriptstyle -1} \, )$} where {\small $( \, \mathrm{K} \hspace{-0.06em}~\doteq~ \sum \, m_i \, f_i \, c^2 \, )$} \hspace{+0.06em}$\rightarrow$\hspace{+0.06em} {\small $( \, \mathbf{V}\mathrm{\scriptscriptstyle{E}} \; \mathrm{E} ~=~ \mathbf{V}\mathrm{\scriptscriptstyle{K}} \; \mathrm{K} \, )$ $( \, \mathbf{V}\mathrm{\scriptscriptstyle{E}} ~=~ \mathbf{V}\mathrm{\scriptscriptstyle{K}} \; \varepsilon \, )$ $( \, \varepsilon ~\doteq~ \mathrm{K}\hspace{+0.09em}/\hspace{+0.12em}\mathrm{E} \, )$} In this paper, we will use : {\small $( \, \mathbf{V}\mathrm{\scriptscriptstyle{E}} \, )$ \hbox {( $ \mathbf{V}\mathrm{\scriptscriptstyle{E}} ~=~ \mathbf{V}$ )}}

\vspace{+0.60em}

\par \bigskip {\centering\subsection*{References \& Bibliography}}\addcontentsline{toc}{subsection}{5. References \& Bibliography}\hypertarget{ref}{}

\bigskip \smallskip

\par \noindent [\,1\,] \textbf{A. Tobla}, A Reformulation of Special Relativity, (2024)\hspace{+0.09em}.\hspace{+0.09em}(\hspace{+0.09em}\href{https://doi.org/10.5281/zenodo.11466228}{\texttt{doi}}\hspace{+0.09em})
\vspace{+1.02em}
\par \noindent [\,2\,] \textbf{A. Torassa}, Special Relativity: Types of Forces, (2024)\hspace{+0.09em}.\hspace{+0.09em}(\hspace{+0.09em}\href{https://doi.org/10.5281/zenodo.14275580}{\texttt{doi}}\hspace{+0.09em})
\vspace{+1.02em}
\par \noindent [\,3\,] \textbf{A. Torassa}, Special Relativity: Center of Mass-Energy, (2025)\hspace{+0.09em}.\hspace{+0.09em}(\hspace{+0.09em}\href{https://doi.org/10.5281/zenodo.15410957}{\texttt{doi}}\hspace{+0.09em})
\vspace{+1.02em}
\par \noindent [\hspace{+0.063em}{\small A}\hspace{+0.063em}] \textbf{C. M{\o}ller}, The Theory of Relativity, (1952)\hspace{+0.09em}.
\vspace{+1.02em}
\par \noindent [\hspace{+0.093em}{\small B}\hspace{+0.093em}] \textbf{W\hspace{-0.18em}. Pauli}, Theory of Relativity, (1958)\hspace{+0.09em}.

\newpage

\par \bigskip {\centering\subsection*{Annex I}}\hypertarget{p3a1}{}

\par \bigskip {\centering\subsection*{System of Particles}}\addcontentsline{toc}{subsection}{Annex I : System of Particles}

\bigskip \smallskip

\noindent In special relativity, the generalized relativistic energy $( \, \mathrm{E} \, )$ the total energy $( \, \mathrm{T} \, )$ the linear momentum $( \, \mathbf{P} \, )$ the rest mass $( \, \mathrm{M}_o \, )$ and the velocity $( \, \mathbf{V} \, )$ of any massive or non-massive system ( of particles ) are given by:
\par \vspace{-0.30em}
\begin{eqnarray}
\mathrm{E} \hspace{-0.06em}~\doteq~ \sum \, m_i \, f_i \, c^2 + \sum \, \mathrm{E_{nki}}
\end{eqnarray}
\vspace{-0.60em}
\begin{eqnarray}
\mathrm{T} \hspace{-0.06em}~\doteq~ \sum \, m_i \, f_i \, c^2 + \sum \, \mathrm{E_{nki}} + \sum \, \mathrm{E_{nke}}
\end{eqnarray}
\vspace{-0.60em}
\begin{eqnarray}
\mathbf{P} \hspace{-0.06em}~\doteq~ \sum \, m_i \, f_i \, \mathbf{v}_i
\end{eqnarray}
\vspace{-0.60em}
\begin{eqnarray}
\mathrm{M}_o^2 \, c^4 \hspace{-0.06em}~\doteq~ \mathrm{E}^2 - \mathbf{P}^2 c^2
\end{eqnarray}
\vspace{-0.60em}
\hypertarget{eve}{}
\begin{eqnarray}
\mathbf{V} \hspace{-0.06em}~\doteq~ \mathbf{P} \: c^2 \, \mathrm{E}^{\scriptscriptstyle -1}
\end{eqnarray}
\par \vspace{+0.60em}
\noindent where $( \, m_i, \: f_i, \: \mathbf{v}_i \, )$ are the intrinsic mass, the relativistic factor and the velocity of the \hbox {\textit{i}-th} massive or non-massive particle of the system, $( \, \sum \, \mathrm{E_{nki}} \, )$ is the total non-kinetic energy of the system which contributes to the rest mass $( \, \mathrm{M}_o \, )$ of the system, $( \, \sum \, \mathrm{E_{nke}} \, )$ is the total non-kinetic energy of the system which does not contribute to the rest mass $( \, \mathrm{M}_o \, )$ of the system, and $( \, c \, )$ is the speed of light in vacuum. Note : in non-massive systems : $( \, \sum \, \mathrm{E_{nki}} ~=~ 0 \, )$
\par \vspace{+0.60em}
\noindent The intrinsic mass $(\hspace{+0.06em}{\mathrm{M}}\hspace{+0.06em})$ and the relativistic factor $(\hspace{+0.06em}{\mathrm{F}}\hspace{+0.06em})$ of a massive system ( composed of massive particles or non-massive particles, or both at the same time ) \hbox {are given by}:
\par \vspace{-0.60em}
\begin{eqnarray}
\mathrm{M} ~\doteq~ \mathrm{M}_o
\end{eqnarray}
\vspace{-0.90em}
\begin{eqnarray}
\mathrm{F} ~\doteq~ \Big ( 1 - \dfrac{\mathbf{V} \cdot \mathbf{V}}{c^2} \hspace{+0.15em} \Big )^{\hspace{-0.24em}-\hspace{+0.03em}1/2}
\end{eqnarray}
\par \vspace{+0.60em}
\noindent where $( \, \mathrm{M}_o \, )$ is the rest mass of the massive system, $( \, \mathbf{V} \, )$ is the velocity of the massive system, and $( \, c \, )$ is the speed of light in vacuum.
\par \vspace{+0.60em}
\noindent The intrinsic mass $(\hspace{+0.06em}{\mathrm{M}}\hspace{+0.06em})$ and the relativistic factor $(\hspace{+0.06em}{\mathrm{F}}\hspace{+0.06em})$ of a non-massive system ( composed only of non-massive particles, all with the same vector velocity $\mathbf{c}$ ) \hbox {are given by}:
\par \vspace{-0.60em}
\begin{eqnarray}
\mathrm{M} ~\doteq~ \dfrac{h \, \kappa \, n}{c^2}
\end{eqnarray}
\vspace{-0.60em}
\begin{eqnarray}
\mathrm{F} ~\doteq~ \frac{1}{\kappa \, n} \, \sum \, \nu_i
\end{eqnarray}
\par \vspace{+0.60em}
\noindent where $( \hspace{+0.33em} h \hspace{+0.33em} )$ is the Planck constant, \hspace{+0.06em}$( \hspace{+0.30em} \nu_i \hspace{+0.30em} )$ is the frequency of the \textit{i}-th non-massive particle of the non-massive system, $( \, \kappa \, )$ is a positive universal constant with dimension of frequency, $( \, n \, )$ is the number of non-massive particles of the non-massive system, and $( \, c \, )$ is the speed of light in vacuum.
\par \vspace{+0.60em}
\noindent According to this paper, a massive system $( \, \mathrm{M}_o \ne 0 \, )$ is a system with non-zero rest mass \hbox {( or a system} whose speed $\mathrm{V}$ in vacuum is less than $c$ ) and a non-massive system $( \, \mathrm{M}_o = 0 \, )$ is a system with zero rest mass ( or a system whose speed $\mathrm{V}$ in vacuum is $c$ )
\par \vspace{+0.60em}
\noindent Note : The rest mass $( \, \mathrm{M}_o \, )$ and the intrinsic mass $( \, \mathrm{M} \, )$ are in general not additive, and the relativistic mass $( \, {\mathtt{M}} \, )$ of a system ( massive or non-massive ) is given by : \hbox {\hspace{-0.09em}( ${\mathtt{M}} ~\doteq~ \mathrm{M} \, \mathrm{F}$ )}

\newpage

\par \bigskip {\centering\subsection*{The Einsteinian Kinematics}}

\bigskip \smallskip

\noindent The special position $( \, \bar{\mathbf{\hspace{+0.12em}R}} \: )$ the special velocity $( \, \bar{\mathbf{V}} \, )$ and the special acceleration $( \, \bar{\mathbf{A}} \, )$ of a system ( massive or non-massive ) are given by:
\par \vspace{-0.30em}
\begin{eqnarray}
\bar{\mathbf{\hspace{+0.12em}R}}\hspace{-0.24em} \hspace{+0.12em}~\doteq~ \int \mathrm{F} \, \mathbf{V} \; d\hspace{+0.012em}t
\end{eqnarray}
\vspace{-0.45em}
\begin{eqnarray}
\bar{\mathbf{V}} \hspace{-0.03em}~\doteq~ \dfrac{d\hspace{-0.09em}\bar{\mathbf{\hspace{+0.12em}R}}}{d\hspace{+0.012em}t} ~=~ \mathrm{F} \, \mathbf{V}
\end{eqnarray}
\vspace{-0.30em}
\begin{eqnarray}
\bar{\mathbf{A}} \hspace{+0.03em}~\doteq~ \dfrac{d\hspace{+0.03em}\bar{\mathbf{V}}}{d\hspace{+0.012em}t} ~=~ \mathrm{F} \, \dfrac{d\hspace{+0.03em}\mathbf{V}}{d\hspace{+0.012em}t} + \dfrac{d\hspace{+0.03em}\mathrm{F}}{d\hspace{+0.012em}t} \, \mathbf{V}
\end{eqnarray}
\par \vspace{+1.20em}
\noindent where $( \, \mathrm{F} \, )$ is the relativistic factor of the system, $( \, \mathbf{V} \, )$ is the velocity of the system, and $( \, t \, )$ is the (\hspace{+0.15em}coordinate\hspace{+0.15em}) time.

\vspace{+0.60em}

\par \bigskip {\centering\subsection*{The Einsteinian Dynamics}}

\bigskip \smallskip

\noindent If we consider a system ( massive or non-massive ) with intrinsic mass $( \, \mathrm{M} \, )$ then the \hbox {linear} momentum $( \, \mathbf{P} \, )$ of the system, the angular momentum $( \, \mathbf{L} \, )$ of the system, the net \hbox {Einsteinian} force $( \, \mathbf{F} \, )$ acting on the system, the work $( \, \mathrm{W} \, )$ done by the net Einsteinian forces acting on the system, the kinetic energy $( \, \mathrm{K} \, )$ of the system, the generalized relativistic energy $( \, \mathrm{E} \, )$ of the system, and the total energy $( \, \mathrm{T} \, )$ of the system, they are given by:
\par \vspace{-0.30em}
\begin{eqnarray}
\mathbf{P} \hspace{-0.06em}~\doteq~ \sum \, \mathbf{p}_i ~=~ \sum \, m_i \, \bar{\mathbf{v}}_i ~=~ \sum \, m_i \, f_i \, \mathbf{v}_i ~=~ \mathrm{M} \, \bar{\mathbf{V}} ~=~ \mathrm{M} \, \mathrm{F} \, \mathbf{V}
\end{eqnarray}
\vspace{-0.30em}
\begin{eqnarray}
\mathbf{L} \hspace{+0.03em}~\doteq~ \sum \, \mathbf{l}_{\hspace{+0.06em}i} ~=~ \sum \, \mathbf{r}_i \times \mathbf{p}_i ~=~ \sum \, m_i \; \mathbf{r}_i \times \bar{\mathbf{v}}_i ~=~ \sum \, m_i \, f_i \: \mathbf{r}_i \times \mathbf{v}_i
\end{eqnarray}
\vspace{-0.30em}
\begin{eqnarray}
\mathbf{F} ~=~ \sum \, \mathbf{f}_{\hspace{+0.06em}i} ~=~ \sum \, \dfrac{d\hspace{+0.045em}\mathbf{p}_i}{d\hspace{+0.012em}t} ~=~ \dfrac{d\hspace{+0.045em}\mathbf{P}}{d\hspace{+0.012em}t} ~=~ \mathrm{M} \, \bar{\mathbf{A}} ~=~ \mathrm{M} \, \bigg [ \, \mathrm{F} \, \dfrac{d\hspace{+0.03em}\mathbf{V}}{d\hspace{+0.012em}t} + \dfrac{d\hspace{+0.03em}\mathrm{F}}{d\hspace{+0.012em}t} \, \mathbf{V} \, \bigg ]
\end{eqnarray}
\vspace{-0.15em}
\begin{eqnarray}
\mathrm{W} \hspace{-0.312em}~\doteq~ \sum \, \int_{\scriptscriptstyle 1}^{\hspace{+0.09em}{\scriptscriptstyle 2}} \mathbf{f}_{\hspace{+0.06em}i} \cdot d\hspace{+0.036em}\mathbf{r}_i ~=~ \sum \, \int_{\scriptscriptstyle 1}^{\hspace{+0.09em}{\scriptscriptstyle 2}} \dfrac{d\hspace{+0.045em}\mathbf{p}_i}{d\hspace{+0.012em}t} \cdot d\hspace{+0.036em}\mathbf{r}_i ~=~ \Delta \, \mathrm{K}
\end{eqnarray}
\vspace{-0.30em}
\begin{eqnarray}
\mathrm{K} \hspace{-0.06em}~\doteq~ \sum \, m_i \, f_i \, c^2
\end{eqnarray}
\vspace{-0.30em}
\begin{eqnarray}
\mathrm{E} \hspace{-0.03em}~\doteq~ \sum \, m_i \, f_i \, c^2 + \sum \, \mathrm{E_{nki}} ~=~ \mathrm{K} + \sum \, \mathrm{E_{nki}} ~=~ \mathrm{M} \, \mathrm{F} \, c^2
\end{eqnarray}
\vspace{-0.30em}
\begin{eqnarray}
\mathrm{T} \hspace{-0.06em}~\doteq~ \sum \, m_i \, f_i \, c^2 + \sum \, \mathrm{E_{nki}} + \sum \, \mathrm{E_{nke}} ~=~ \mathrm{M} \, \mathrm{F} \, c^2 + \sum \, \mathrm{E_{nke}}
\end{eqnarray}
\par \vspace{+0.90em}
\noindent where $( \: m_i, \: f_i, \: \mathbf{r}_i, \: \mathbf{v}_i, \: \bar{\mathbf{v}}_i \: )$ are the intrinsic mass, the relativistic factor, the position, the velocity and the special velocity of the \textit{i}-th massive or non-massive particle of the system, $( \: \mathrm{F}, \: \mathbf{V}, \: \bar{\mathbf{V}}, \: \bar{\mathbf{A}} \: )$ are the relativistic factor, the velocity, the special velocity and the special acceleration of the system, $( \, \sum \, \mathrm{E_{nki}} \, )$ is the total non-kinetic energy of the system which contributes to the rest mass $( \, \mathrm{M}_o \, )$ of the system, $( \, \sum \, \mathrm{E_{nke}} \, )$ is the total non-kinetic energy of the system which does not contribute to the rest mass $( \, \mathrm{M}_o \, )$ of the system, $( \, t \, )$ is the \hbox {(\hspace{+0.15em}coordinate\hspace{+0.15em})} time, and $( \, c \, )$ is the speed of light in vacuum. Note : (\hspace{+0.24em}$\sum \, \mathrm{E_{nki}} \,=\, 0$\hspace{+0.24em}) in massive or non-massive particle \hspace{+0.36em}$\rightarrow$\hspace{+0.36em} (\hspace{+0.24em}$\mathrm{E} \,=\, \mathrm{K}$\hspace{+0.24em}) in massive or non-massive particle $|$ Alternative ( first criterion ) : $\mathrm{K} \hspace{-0.06em}~\doteq~ \sum \, (\hspace{+0.15em}m_i \, f_i \, c^2 - m_{oi} \, c^2\hspace{+0.15em})$ \hspace{+0.15em},\hspace{+0.15em} $\mathrm{E}_{\dot{o}} \hspace{-0.06em}~\doteq~ \sum \, m_{oi} \, c^2$ \hspace{+0.15em},\hspace{+0.15em} $\mathrm{E} \hspace{-0.06em}~\doteq~ \mathrm{K} + \mathrm{E}_{\dot{o}}$ \hspace{+0.15em},\hspace{+0.15em} $\mathrm{G} \hspace{-0.06em}~\doteq~ \mathrm{E} + \sum \, \mathrm{E_{nri}} ~=~ \mathrm{M} \, \mathrm{F} \, c^2$

\newpage

\par \bigskip {\centering\subsection*{Annex II}}\hypertarget{p3a2}{}

\par \bigskip {\centering\subsection*{The Kinetic Forces}}\addcontentsline{toc}{subsection}{Annex II : The Kinetic Forces}

\bigskip \smallskip

\noindent The kinetic force \hbox {$\mathbf{K}^{a}_{\hspace{+0.012em}ij}$ exerted} on a particle $i$ with intrinsic mass $m_i$ by another particle $j$ with intrinsic mass $m_j$, \hbox {is given by}:
\par \vspace{-0.54em}
\begin{eqnarray}
\mathbf{K}^{a}_{\hspace{+0.012em}ij} \,=\, - \; \Bigg [ \; \dfrac{m_i \, m_j}{\mathbb{M}} \, ( \, \bar{\mathbf{a}}_{\hspace{+0.045em}i} \hspace{+0.045em}-\, \bar{\mathbf{a}}_{j} \, ) \; \Bigg ]
\end{eqnarray}
\par \vspace{+0.60em}
\noindent where $\bar{\mathbf{a}}_{\hspace{+0.045em}i}$ is the special acceleration of particle $i$, $\bar{\mathbf{a}}_{j}$ is the special acceleration of particle $j$ and $\mathbb{M}$ {\small ( $ = \sum_z^{\scriptscriptstyle{\mathit{All}}} m_z$ )} is the sum of the intrinsic masses of all the particles of the Universe.
\par \vspace{+0.60em}
\noindent On the other hand, the kinetic force $\mathbf{K}^{u}_{\hspace{+0.030em}i}$ exerted on a particle $i$ with intrinsic mass $m_i$ by the Universe, is given by:
\par \vspace{-0.45em}
\begin{eqnarray}
\mathbf{K}^{u}_{\hspace{+0.030em}i} \,=\, - \; m_i \; \dfrac{\sum_z^{\scriptscriptstyle{\mathit{All}}} m_z \, \bar{\mathbf{a}}_{\hspace{+0.045em}z}}{\sum_z^{\scriptscriptstyle{\mathit{All}}} m_z}
\end{eqnarray}
\par \vspace{+0.60em}
\noindent where $m_z$ and $\bar{\mathbf{a}}_{\hspace{+0.045em}z}$ are the intrinsic mass and the special acceleration of the \textit{z}-th particle of the Universe.
\par \vspace{+0.60em}
\noindent From the above equations it follows that the net kinetic force $\mathbf{K}_i$ {\small ( $ = \sum_j^{\scriptscriptstyle{\mathit{All}}} \, \mathbf{K}^{a}_{\hspace{+0.012em}ij}$} {\small $+ \; \mathbf{K}^{u}_{\hspace{+0.030em}i}$ )} acting on a particle $i$ with intrinsic mass $m_i$, is given by:
\par \vspace{-0.60em}
\begin{eqnarray}
\mathbf{K}_i \,=\, - \, m_i \, \bar{\mathbf{a}}_{\hspace{+0.045em}i}
\end{eqnarray}
\par \vspace{+0.60em}
\noindent where $\bar{\mathbf{a}}_{\hspace{+0.045em}i}$ is the special acceleration of particle $i$.
\par \vspace{+0.60em}
\noindent Now, from the Einsteinian dynamics $[ \, \hyperlink{efe}{10} \, ]$ we have:
\par \vspace{-0.60em}
\begin{eqnarray}
\mathbf{F}_i \,=\, m_i \, \bar{\mathbf{a}}_{\hspace{+0.045em}i}
\end{eqnarray}
\par \vspace{+0.60em}
\noindent Since {\small (\hspace{+0.240em}$\mathbf{K}_i \,=\, - \, m_i \, \bar{\mathbf{a}}_{\hspace{+0.045em}i}$\hspace{+0.240em})} we obtain:
\par \vspace{-0.81em}
\begin{eqnarray}
\mathbf{F}_i \,=\, - \, \mathbf{K}_i
\end{eqnarray}
\par \vspace{+0.30em}
\noindent that is:
\par \vspace{-0.81em}
\begin{eqnarray}
\mathbf{K}_i \, + \, \mathbf{F}_i \,=\, 0
\end{eqnarray}
\par \vspace{+0.60em}
\noindent If {\small (\hspace{+0.240em}$\mathbf{T}_i \,\doteq\, \mathbf{K}_i \, + \, \mathbf{F}_i$\hspace{+0.240em})} then:
\par \vspace{-0.81em}
\begin{eqnarray}
\mathbf{T}_i \,=\, 0
\end{eqnarray}
\par \vspace{+0.60em}
\noindent Therefore, if the net kinetic force $\mathbf{K}_i$ is added in the Einsteinian dynamics then the total \hbox {force $\mathbf{T}_i$} acting on a ( massive or non-massive ) particle $i$ is always zero.
\par \vspace{+0.21em}
\noindent Note : According to this paper, the kinetic forces ${\stackrel{\scriptstyle{au}}{\smash{\mathbf{K}}\rule{0pt}{+0.63em}}}$ are directly related to kinetic energy $\mathrm{K}$.

\end{document}
