
\documentclass[10pt,fleqn]{article}
%\documentclass[a4paper,10pt]{article}�
%\documentclass[letterpaper,10pt]{article}

%\usepackage[dvips]{geometry}
%\geometry{papersize={162.0mm,234.0mm}}
%\geometry{totalwidth=141.0mm,totalheight=198.0mm}
\usepackage[papersize={162.0mm,261.0mm},totalwidth=141.0mm,totalheight=225.0mm]{geometry}

\usepackage[english]{babel}
\usepackage[latin1]{inputenc}
%\usepackage[utf8]{inputenc}
%\usepackage[T1]{fontenc}
%\usepackage{ae,aecompl}
%\usepackage{mathptmx}
%\usepackage{pslatex}
%\usepackage{lmodern}
\usepackage{amsfonts}
\usepackage{amsmath}
\usepackage{amssymb}

%\frenchspacing

\usepackage{hyperref}
\hypersetup{colorlinks=true,linkcolor=black,urlcolor=blue,bookmarksopen=true}
\hypersetup{bookmarksnumbered=true,pdfstartview=FitH,pdfpagemode=UseOutlines}
%\hypersetup{bookmarksnumbered=true,pdfstartview=FitH,pdfpagemode=UseNone}
\hypersetup{pdftitle={On Fields, Gravitational Mass and Inertial Mass}}
\hypersetup{pdfauthor={Alejandro A. Torassa}}

\setlength{\arraycolsep}{1.74pt}

\newcommand{\vet}{\medskip}
\newcommand{\vek}{\bigskip \bigskip}

%\ccl Attribution 4.0 International License
%\cct Attribution 4.0 International License
%\newcommand{\ccl}{${\large{\textcircled{\raisebox{+0.8pt}{{\scriptsize{{\textsf{cc}}}}}}}}$\;}
%\newcommand{\cct}{${\normalsize{\textcircled{\raisebox{+1.00pt}{{\tiny{{\texttt{cc}}}}}}}}$\;}

\newcommand{\med}{\raise.5ex\hbox{$\scriptstyle 1$}\kern-.15em/\kern-.09em\lower.25ex\hbox{$\scriptstyle 2$}}
\newcommand{\met}{\raise.5ex\hbox{$\scriptstyle 1$}\kern-.15em/\kern-.09em\lower.25ex\hbox{$\scriptstyle 2$}}

\begin{document}

\enlargethispage{+0.00em}

\addcontentsline{toc}{section}{On Fields, Gravitational Mass and Inertial Mass}

\begin{center}

{\fontsize{13}{13}\selectfont \sc On Fields, Gravitational Mass and Inertial Mass}

\vek

{\fontsize{10.80}{10.80}\selectfont Alejandro A. Torassa}

\vek

\small

Creative Commons Attribution 4.0 License

\vet

\href{https://orcid.org/0000-0002-1389-247X}{\scriptsize{ORCID}} \S \ (2025) Buenos Aires
%\href{https://orcid.org/0000-0002-1389-247X}{\normalsize{\textsc{orcid}}} \S \ (2025) Buenos Aires

\vet

Argentina

\bigskip \bigskip

\parbox{105,00mm}{In classical mechanics, this paper presents a new scalar field that explains the relationship between gravitational mass and inertial mass. According to this paper, inertial mass is determined by gravitational mass. In classical dynamics, gravitational mass (inertial mass) is more relevant than the other charges (electric charge, etc.) because it is always positive (this property is very important in the new scalar field)}

\end{center}

\normalsize

\vspace{-1.20em}

\par \bigskip {\centering\subsection*{Introduction}}\addcontentsline{toc}{subsection}{1. Introduction}

\bigskip \smallskip

\noindent In classical mechanics, this paper considers that there is a type of scalar field that basically depends on the scalar value of the charges but does not depend at all on the position, velocity, acceleration, etc. of the charges.
\par \vspace{+0.66em}
\noindent Generically, the new scalar field $( \, \mathit{C} \, )$ produced by a generic charge, would be given by:
\par \vspace{+0.90em}
\par \hspace{+0.30em} $\mathit{C} ~=~ \mathit{K} \, \mathit{c}$ \hfill (1)
\par \vspace{+0.90em}
\par \noindent where $( \, \mathit{K} \, )$ is a universal field constant, and $( \, \mathit{c} \, )$ is the scalar value of the generic charge.
\par \vspace{+0.66em}
\noindent Therefore, the new scalar field $( \, \mathit{C}{\hspace{-0.12em}\scriptscriptstyle{\mathit E}} \, )$ produced by an electric charge and the new scalar field $( \, \mathit{C}{\hspace{-0.12em}\scriptscriptstyle{\mathit G}} \, )$ produced by a gravitational mass, would be given by:
\par \vspace{+0.90em}
\par \hspace{+0.30em} $\mathit{C}{\hspace{-0.12em}\scriptscriptstyle{\mathit E}} ~=~ \mathit{K}{\hspace{-0.12em}\scriptscriptstyle{\mathit E}} \:\, \mathit{e}$ \hfill (2)
\par \vspace{+0.90em}
\par \hspace{+0.30em} $\mathit{C}{\hspace{-0.12em}\scriptscriptstyle{\mathit G}} ~=~ \mathit{K}{\hspace{-0.12em}\scriptscriptstyle{\mathit G}} \:\, \hat{\mathit{m}}$ \hfill (3)
\par \vspace{+0.90em}
\par \noindent where $( \, \mathit{K}{\hspace{-0.12em}\scriptscriptstyle{\mathit E}} \, )$ is a new universal electric constant, $( \, \mathit{e} \, )$ is the value of the electric charge, $( \, \mathit{K}{\hspace{-0.12em}\scriptscriptstyle{\mathit G}} \, )$ is a new universal gravitational constant, and $( \, \hat{\mathit{m}} \, )$ is the value of the gravitational mass.
\par \vspace{+0.66em}
\noindent Additionally, the extra force $( \, \mathbf{F}_{\scriptscriptstyle{\mathrm A}} \, )$ acting on a generic charge A, due to the new total scalar field $( \, \mathit{C}_{\scriptscriptstyle{\mathrm T}} \, )$ produced by all generic charges, would be given by:
\par \vspace{+0.90em}
\par \hspace{+0.30em} $\mathbf{F}_{\scriptscriptstyle{\mathrm A}} ~=\, - \, \mathit{c}_{\scriptscriptstyle{\mathrm A}} \; \mathbf{a}_{\scriptscriptstyle{\mathrm A}} \: \mathit{C}_{\scriptscriptstyle{\mathrm T}} ~=\, - \, \mathit{c}_{\scriptscriptstyle{\mathrm A}} \; \mathbf{a}_{\scriptscriptstyle{\mathrm A}} \; \mathit{K} \: \sum^{\scriptscriptstyle{All}}_i \mathit{c}_i$ \hfill (4)
\par \vspace{+0.90em}
\par \noindent where $( \, \mathit{c}_{\scriptscriptstyle{\mathrm A}} \, )$ is the value of the generic charge A, and $( \, \mathbf{a}_{\scriptscriptstyle{\mathrm A}} \, )$ is the ordinary acceleration of the \hbox {generic charge A.}
\par \vspace{+0.66em}
\noindent Therefore, the additional force $( \, \mathbf{F}{\scriptscriptstyle{\mathrm E}}_{\scriptscriptstyle{\mathrm A}} \, )$ acting on an electric charge A, due to the new total scalar field $( \, \mathit{C}{\hspace{-0.12em}\scriptscriptstyle{\mathit E}}_{\scriptscriptstyle{\mathit T}} \, )$ produced by all the electric charges, and the additional force $( \, \mathbf{F}{\scriptscriptstyle{\mathrm G}}_{\scriptscriptstyle{\mathrm A}} \, )$ acting on a gravitational mass A, due to the new total scalar field $( \, \mathit{C}{\hspace{-0.12em}\scriptscriptstyle{\mathit G}}_{\scriptscriptstyle{\mathit T}} \, )$ produced by all gravitational masses, would be given by:
\par \vspace{+0.90em}
\par \hspace{+0.30em} $\mathbf{F}{\scriptscriptstyle{\mathrm E}}_{\scriptscriptstyle{\mathrm A}} ~=\, - \, \mathit{e}_{\scriptscriptstyle{\mathrm A}} \; \mathbf{a}_{\scriptscriptstyle{\mathrm A}} \: \mathit{C}{\hspace{-0.12em}\scriptscriptstyle{\mathit E}}_{\scriptscriptstyle{\mathit T}} ~=\, - \, \mathit{e}_{\scriptscriptstyle{\mathrm A}} \; \mathbf{a}_{\scriptscriptstyle{\mathrm A}} \; \mathit{K}{\hspace{-0.12em}\scriptscriptstyle{\mathit E}} \: \sum^{\scriptscriptstyle{All}}_i \mathit{e}_i$ \hfill (5)
\par \vspace{+1.20em}
\par \hspace{+0.30em} $\mathbf{F}{\scriptscriptstyle{\mathrm G}}_{\scriptscriptstyle{\mathrm A}} ~=\, - \, \hat{\mathit{m}}_{\scriptscriptstyle{\mathrm A}} \; \mathbf{a}_{\scriptscriptstyle{\mathrm A}} \: \mathit{C}{\hspace{-0.12em}\scriptscriptstyle{\mathit G}}_{\scriptscriptstyle{\mathit T}} ~=\, - \, \hat{\mathit{m}}_{\scriptscriptstyle{\mathrm A}} \; \mathbf{a}_{\scriptscriptstyle{\mathrm A}} \; \mathit{K}{\hspace{-0.12em}\scriptscriptstyle{\mathit G}} \: \sum^{\scriptscriptstyle{All}}_i \hat{\mathit{m}}_i$ \hfill (6)
\par \vspace{+0.90em}
\par \noindent where $( \hspace{+0.09em} \mathit{e}_{\scriptscriptstyle{\mathrm A}} \hspace{+0.09em} )$ is the value of the electric charge A, $( \hspace{+0.09em} \hat{\mathit{m}}_{\scriptscriptstyle{\mathrm A}} \hspace{+0.09em} )$ is the value of the gravitational mass A, and $( \hspace{+0.09em} \mathbf{a}_{\scriptscriptstyle{\mathrm A}} \hspace{+0.09em} )$ is the ordinary acceleration of the electric charge A and the \hbox {gravitational mass A.}

\newpage

\par \bigskip {\centering\subsection*{General Observations}}\addcontentsline{toc}{subsection}{2. General Observations}

\bigskip \smallskip

\noindent Since electric charges are balanced in the Universe ( negative charges and positive charges are always in equilibrium ) then the new total scalar field $( \, \mathit{C}{\hspace{-0.12em}\scriptscriptstyle{\mathit E}}_{\scriptscriptstyle{\mathit T}} \, )$ is always zero and therefore the additional force $( \, \mathbf{F}{\scriptscriptstyle{\mathrm E}}_{\scriptscriptstyle{\mathrm A}} \, )$ acting on any electric charge A is also always zero.
\par \bigskip
\noindent However, gravitational masses are not balanced in the Universe ( because gravitational masses are always positive ) so the new total scalar field $( \, \mathit{C}{\hspace{-0.12em}\scriptscriptstyle{\mathit G}}_{\scriptscriptstyle{\mathit T}} \, )$ is never zero and therefore the additional force $( \, \mathbf{F}{\scriptscriptstyle{\mathrm G}}_{\scriptscriptstyle{\mathrm A}} \, )$ acting on any gravitational mass A is not always zero.
\par \bigskip
\noindent Therefore, the additional force $( \, \mathbf{F}{\scriptscriptstyle{\mathrm E}}_{\scriptscriptstyle{\mathrm A}} \, )$ acting on any electric charge A should not be added to the Lorentz force ( because the force $\mathbf{F}{\scriptscriptstyle{\mathrm E}}_{\scriptscriptstyle{\mathrm A}}$ \hbox {is always zero )}
\par \bigskip
\noindent However, the additional force $( \, \mathbf{F}{\scriptscriptstyle{\mathrm G}}_{\scriptscriptstyle{\mathrm A}} \, )$ acting on any gravitational mass A must be added to Newton's gravitational force ( because the force $\mathbf{F}{\scriptscriptstyle{\mathrm G}}_{\scriptscriptstyle{\mathrm A}}$ is not always zero )
\par \bigskip
\noindent The magnetic force component of the Lorentz force involves the cross product between the velocity of the electric charge and the magnetic field $( \, \mathbf{v} \times \mathbf{B} \, )$ but the additional gravitational force component of Newton's gravitational force involves the product between the acceleration of the gravitational mass and the new total gravitational scalar field $( \, \mathbf{a} \: \mathit{C} \, )$
\par \bigskip
\noindent Therefore, Newton's second law now takes the following form : $( \, \sum \, \mathbf{F}_{\scriptscriptstyle{\mathrm A}} \,=\, 0 \, )$ and the inertial mass $( \, \mathit{m}_{\scriptscriptstyle{\mathrm A}} \, )$ of any particle A is now given by the following equation : $( \, \mathit{m}_{\scriptscriptstyle{\mathrm A}} \,=\, \hat{\mathit{m}}_{\scriptscriptstyle{\mathrm A}} \; \mathit{K}{\hspace{-0.12em}\scriptscriptstyle{\mathit G}} \: \sum^{\scriptscriptstyle{All}}_i \hat{\mathit{m}}_i \, )$ ( this reformulation is now consistent with experience )
\par \bigskip
\noindent In other words, Newton's second law states : $( \, \sum \, \mathbf{F}_{\scriptscriptstyle{\mathrm A}} \:=\: \mathit{m}_{\scriptscriptstyle{\mathrm A}} \, \mathbf{a}_{\scriptscriptstyle{\mathrm A}} \, )$ then $( \, \sum \, \mathbf{F}_{\scriptscriptstyle{\mathrm A}} -\, \mathit{m}_{\scriptscriptstyle{\mathrm A}} \, \mathbf{a}_{\scriptscriptstyle{\mathrm A}} \:=\: 0 \, )$ $\rightarrow$ $( \, \sum \, \mathbf{F}_{\scriptscriptstyle{\mathrm A}} +\, \mathbf{F}{\scriptscriptstyle{\mathrm G}}_{\scriptscriptstyle{\mathrm A}} \:=\: 0 \, )$ $\rightarrow$ $( \, \sum \, \mathbf{F}_{\scriptscriptstyle{\mathrm A}} \,=\, 0 \, )$ . Since the additional gravitational force $( \, \mathbf{F}{\scriptscriptstyle{\mathrm G}}_{\scriptscriptstyle{\mathrm A}} \, )$ must be added to Newton's gravitational force.
\par \bigskip
\noindent Additionally, according to this paper, the additional gravitational force $( \, \mathbf{F}{\scriptscriptstyle{\mathrm G}}_{\scriptscriptstyle{\mathrm A}} \, )$ is related to the force of inertia $( \, -\, \mathit{m}_{\scriptscriptstyle{\mathrm A}} \, \mathbf{a}_{\scriptscriptstyle{\mathrm A}} \, )$ ( vis insita ) and the new gravitational scalar field $( \, \mathit{C}{\hspace{-0.12em}\scriptscriptstyle{\mathit G}} \, )$ is related to Mach's principle ( as the origin of $-\, \mathit{m}_{\scriptscriptstyle{\mathrm A}} \, \mathbf{a}_{\scriptscriptstyle{\mathrm A}}$ )
\par \bigskip
\noindent On the other hand, if the gravitational mass is intrinsic then according to the equation above $( \, \mathit{m}_{\scriptscriptstyle{\mathrm A}} \,=\, \hat{\mathit{m}}_{\scriptscriptstyle{\mathrm A}} \; \mathit{K}{\hspace{-0.12em}\scriptscriptstyle{\mathit G}} \: \sum^{\scriptscriptstyle{All}}_i \hat{\mathit{m}}_i \, )$ the inertial mass is intrinsic ( since the new gravitational scalar field $\mathit{C}{\hspace{-0.12em}\scriptscriptstyle{\mathit G}}$ is intrinsic too )
\par \bigskip
\noindent Therefore, the gravitational mass, the inertial mass and the new gravitational scalar field are invariant under transformations between inertial and non-inertial reference frames.
\par \bigskip
\noindent Finally, if the ordinary acceleration $( \, \mathbf{a}_{\scriptscriptstyle{\mathrm A}} \, )$ is expressed with universal magnitudes [\,1\,] or inertial magnitudes [\,2\,] ( relational magnitudes ) then the gravitational force $( \, \mathbf{F}{\scriptscriptstyle{\mathrm G}}_{\scriptscriptstyle{\mathrm A}} \, )$ would also be invariant under transformations between inertial and non-inertial reference frames.

\vspace{+0.60em}

\par \bigskip {\centering\subsection*{References \& Bibliography}}\addcontentsline{toc}{subsection}{3. References \& Bibliography}

\bigskip \smallskip

\par \noindent [\,1\,] \textbf{A. Tobla}, A Reformulation of Classical Mechanics ( I \& II ), (2024)\hspace{+0.09em}.\hspace{+0.09em}(\hspace{+0.09em}\href{https://doi.org/10.5281/zenodo.11207437}{\texttt{doi}}\hspace{+0.09em})
\bigskip \smallskip
\par \noindent [\,2\,] \textbf{A. Tobla}, A Reformulation of Classical Mechanics ( III \& IV ), (2024)\hspace{+0.09em}.\hspace{+0.09em}(\hspace{+0.09em}\href{https://doi.org/10.5281/zenodo.11207459}{\texttt{doi}}\hspace{+0.09em})
\bigskip \smallskip
\par \noindent [\hspace{+0.063em}{\small A}\hspace{+0.063em}] \textbf{E. Mach}, The Science of Mechanics, (1883)\hspace{+0.09em}.

\newpage

\par \bigskip {\centering\subsection*{Appendix}}

\par \bigskip {\centering\subsection*{Relativistic Scalar Field}}\addcontentsline{toc}{subsection}{4. Appendix : Relativistic Scalar Field}

\bigskip \smallskip

\noindent In special relativity ( possibly ) the new gravitational scalar field $( \, \mathit{C}{\hspace{-0.12em}\scriptscriptstyle{\mathit G}} \, )$ produced by a massive or non-massive particle with energetic charge, for the inertial reference frame in which the Universe has no translation or rotation, would be given by:
\par \vspace{+0.81em}
\par \hspace{+0.30em} $\mathit{C}{\hspace{-0.12em}\scriptscriptstyle{\mathit G}} ~=~ \mathit{K}{\hspace{-0.12em}\scriptscriptstyle{\mathit G}} \:\, ( \, \mathit{E} + \hspace{+0.24em}\bar{\hspace{-0.24em}\mathit{E}} \, ) \hspace{+0.15em}/\hspace{+0.15em} c^2$ \hfill (1)
\par \vspace{+0.81em}
\par \noindent where $( \, \mathit{K}{\hspace{-0.12em}\scriptscriptstyle{\mathit G}} \, )$ is a new universal gravitational constant, $( \, \mathit{E} \, )$ is the relativistic energy of the particle, and $( \, \hspace{+0.24em}\bar{\hspace{-0.24em}\mathit{E}} \, )$ is the non-relativistic energy of the particle [\,1\,]
\par \vspace{+0.81em}
\noindent Additionally, now the extra gravitational force $( \, \mathbf{F}{\scriptscriptstyle{\mathrm G}}_{\scriptscriptstyle{\mathrm A}} \, )$ acting on a massive or non-massive particle A, due to the new total gravitational scalar field $( \, \mathit{C}{\hspace{-0.12em}\scriptscriptstyle{\mathit G}}_{\scriptscriptstyle{\mathit T}} \, )$ produced by all massive and non-massive particles with energetic charge, would be given by:
\par \vspace{+0.81em}
\par \hspace{+0.30em} $\mathbf{F}{\scriptscriptstyle{\mathrm G}}_{\scriptscriptstyle{\mathrm A}} ~=\, - \, \mathit{E}_{\scriptscriptstyle{\mathrm A}} \; \mathbf{a}_{\scriptscriptstyle{\mathrm A}} \cdot \mathbf{M}^{\scriptscriptstyle{-1}}_{\scriptscriptstyle{\mathrm A}} \: \mathit{C}{\hspace{-0.12em}\scriptscriptstyle{\mathit G}}_{\scriptscriptstyle{\mathit T}} ~=\, - \, \mathit{E}_{\scriptscriptstyle{\mathrm A}} \; \mathbf{a}_{\scriptscriptstyle{\mathrm A}} \cdot \mathbf{M}^{\scriptscriptstyle{-1}}_{\scriptscriptstyle{\mathrm A}} \; \mathit{K}{\hspace{-0.12em}\scriptscriptstyle{\mathit G}} \: \sum^{\scriptscriptstyle{All}}_i ( \, \mathit{E}_i + \hspace{+0.24em}\bar{\hspace{-0.24em}\mathit{E}}_{\hspace{-0.12em}{i}} \, ) \hspace{+0.15em}/\hspace{+0.15em} c^2$ \hfill (2)
\par \vspace{+0.81em}
\par \hspace{+0.30em} $\mathrm{m}_{\scriptscriptstyle{\mathrm A}} ~=~ \mathit{E}_{\scriptscriptstyle{\mathrm A}} \; \mathit{K}{\hspace{-0.12em}\scriptscriptstyle{\mathit G}} \: \sum^{\scriptscriptstyle{All}}_i ( \, \mathit{E}_i + \hspace{+0.24em}\bar{\hspace{-0.24em}\mathit{E}}_{\hspace{-0.12em}{i}} \, ) \hspace{+0.15em}/\hspace{+0.15em} c^2$ \hfill (3)
\par \vspace{+0.81em}
\par \hspace{+0.30em} $\mathbf{F}{\scriptscriptstyle{\mathrm G}}_{\scriptscriptstyle{\mathrm A}} ~=\, - \, \mathrm{m}_{\scriptscriptstyle{\mathrm A}} \; \mathbf{a}_{\scriptscriptstyle{\mathrm A}} \cdot \mathbf{M}^{\scriptscriptstyle{-1}}_{\scriptscriptstyle{\mathrm A}} ~=\, - \, \mathit{m}_{\scriptscriptstyle{\mathrm A}} \, \bar{\mathbf{a}}_{\scriptscriptstyle{\mathrm A}}$ \hfill (4)
\par \vspace{+0.81em}
\par \noindent where $( \, \mathit{E}_{\scriptscriptstyle{\mathrm A}} \, )$ is the relativistic energy of particle A, $( \, \mathrm{m}_{\scriptscriptstyle{\mathrm A}} \, )$ is the relativistic mass of \hbox {particle A} $( \, \mathit{m}_{\scriptscriptstyle{\mathrm A}} \, )$ is the intrinsic mass of particle A, $( \, \mathbf{a}_{\scriptscriptstyle{\mathrm A}} \, )$ is the ordinary acceleration of particle A, $( \, \bar{\mathbf{a}}_{\scriptscriptstyle{\mathrm A}} \, )$ is the special acceleration of particle A, and $( \, \mathbf{M} \, )$ is the M{\o}ller tensor [\,2\,]
\par \vspace{+0.81em}
\noindent On the other hand, the net Einsteinian force $( \, \mathbf{F}\mathrm{\scriptstyle{E}}_{\scriptscriptstyle{\mathrm A}} \, )$ [\,2\,] acting on a massive or non-massive particle A, is given by:
\par \vspace{+0.81em}
\par \hspace{+0.30em} $\mathbf{F}{\scriptstyle{\mathrm E}}_{\scriptscriptstyle{\mathrm A}} ~=~ d\hspace{+0.045em}\mathbf{P}_{\scriptscriptstyle{\mathrm A}} \hspace{+0.15em}/\hspace{+0.15em} d\hspace{+0.012em}t$ \hfill (5)
\par \vspace{+0.81em}
\par \hspace{+0.30em} $\mathbf{F}{\scriptstyle{\mathrm E}}_{\scriptscriptstyle{\mathrm A}} ~=~ \mathit{m}_{\scriptscriptstyle{\mathrm A}} \, \bar{\mathbf{a}}_{\scriptscriptstyle{\mathrm A}}$ \hfill (6)
\par \vspace{+0.81em}
\par \noindent where $( \, \mathit{m}_{\scriptscriptstyle{\mathrm A}} \, )$ is the intrinsic mass of particle A, $( \, \bar{\mathbf{a}}_{\scriptscriptstyle{\mathrm A}} \, )$ is the special acceleration of particle A $( \, \mathbf{P}_{\scriptscriptstyle{\mathrm A}} \, )$ is the linear momentum of particle A, and $( \, t \, )$ is the (\hspace{+0.15em}coordinate\hspace{+0.15em}) time.
\par \vspace{+0.81em}
\noindent Now, since the additional gravitational force $( \, \mathbf{F}{\scriptscriptstyle{\mathrm G}}_{\scriptscriptstyle{\mathrm A}} \, )$ is also another force of Nature then it must be added to the net Einsteinian force $( \, \mathbf{F}\mathrm{\scriptstyle{E}}_{\scriptscriptstyle{\mathrm A}} \, )$ However, the additional gravitational \hbox {force $( \, \mathbf{F}{\scriptscriptstyle{\mathrm G}}_{\scriptscriptstyle{\mathrm A}} \, )$} must be added to the net Einsteinian force $( \, \mathbf{F}\mathrm{\scriptstyle{E}}_{\scriptscriptstyle{\mathrm A}} \, )$ as follows:
\par \vspace{+0.90em}
\par \hspace{+0.30em} $\mathbf{F}{\scriptstyle{\mathrm E}}_{\scriptscriptstyle{\mathrm A}} - \, \mathit{m}_{\scriptscriptstyle{\mathrm A}} \, \bar{\mathbf{a}}_{\scriptscriptstyle{\mathrm A}} ~=~ 0$ \hfill (7)
\par \vspace{+0.81em}
\par \hspace{+0.30em} $\mathbf{F}{\scriptstyle{\mathrm E}}_{\scriptscriptstyle{\mathrm A}} + \mathbf{F}{\scriptscriptstyle{\mathrm G}}_{\scriptscriptstyle{\mathrm A}} ~=~ 0$ \hfill (8)
\par \vspace{+0.81em}
\par \hspace{+0.30em} $\mathbf{F}{\scriptstyle{\mathrm E}}_{\scriptscriptstyle{\mathrm A}} ~=~ 0$ \hfill (9)
\par \vspace{+0.90em}
\noindent Note : the new gravitational scalar field extends throughout the Universe ( without delay )

\vspace{+0.60em}

\par \bigskip {\centering\subsection*{References \& Bibliography}}\addcontentsline{toc}{subsection}{5. Appendix : References \& Bibliography}

\bigskip \smallskip

\par \noindent [\,1\,] \textbf{A. Torassa}, Special Relativity: Types of Energy, (2025)\hspace{+0.09em}.\hspace{+0.09em}(\hspace{+0.09em}\href{https://doi.org/10.5281/zenodo.15535968}{\texttt{doi}}\hspace{+0.09em})
\bigskip \smallskip
\par \noindent [\,2\,] \textbf{A. Torassa}, Special Relativity: Types of Forces, (2024)\hspace{+0.09em}.\hspace{+0.09em}(\hspace{+0.09em}\href{https://doi.org/10.5281/zenodo.14275580}{\texttt{doi}}\hspace{+0.09em})
\bigskip \smallskip
\par \noindent [\hspace{+0.063em}{\small A}\hspace{+0.063em}] \textbf{C. M{\o}ller}, The Theory of Relativity, (1952)\hspace{+0.09em}.

\end{document}
